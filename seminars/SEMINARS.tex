\documentclass[11pt,a4paper]{article}
% \usepackage{natbib}

\usepackage[utf8]{inputenc}
\usepackage{mathpazo}
\usepackage{amsmath,amsfonts,amssymb,amsthm}
\usepackage{mathtools}
\usepackage{todonotes}

\usepackage[margin=3cm]{geometry}


\newcommand{\todoin}[2][]{\todo[inline, color=blue!40, #1]{#2}}
\newcommand{\fixme}[2][]{\todo[color=yellow!40, #1]{#2}}
\newcommand{\fixmein}[2][]{\todo[inline, color=yellow!40, #1]{#2}}


\title{INFO-F-530 - Computer Science Seminar}
\author{Université Libre de Bruxelles \\
Département d'informatique \\
\\ Hakim Boulahya}


\begin{document}

\maketitle
\tableofcontents

\newpage

\begin{abstract}
\end{abstract}

\section{Real-Time Data Mining}
\paragraph{Seminar details}

\begin{itemize}
  \item \textbf{Title:} Real-Time Data Mining
  \item \textbf{Date & Location:} 29/03/2019, Brussels
  \item \textbf{Presenter:} João Gama (INESC TEC, University of Porto)
\end{itemize}

\subsection{Outline}

\paragraph{}

The talk discuss one of the major problem of our century: the exponential growth of data with a focus of real-time consumption of those. This problem is often known as Big Data and the objective of the talk was the discuss a subproblem \todo{sure about saying subproblem} of Big Data: building decision models based on real-time data analysis.

Such a problem a usually approach in a static manneer, that is generating a model based on an existing finite data set. One of the emphasize of the talk is that this learning methodology is not adapted to the world problems: static solution can hardly be used overtime for dynamic models.

To discuss the problem of real time decision model implementation, the presenter introduced a use case: a network of sensors to monitor electrical power supply. The methodology proposed to build the model is by using the Only Divisive-Agglomerative Clustering (ODAC) \todo{add ref.}, to maintain a continuous cluster structure from real-time data. This structure is build dynamically by using different operations: split and merge. Those operations enables the cluster to be expanded in two ways when changes are required, respectively: more details in the cluster by splitting nodes or merging substructures. Splist or merges of the clusterapplied in a probabilistic maneer using Hoeffding bounds \todo{add ref}, mathematical bounds based on confidence level and observed data streams.

Finally, the presenter conclude the talk by discussing the available tool for data streams analysis and a overview of a generic model for online adaptive learning algorithms.
\fixmein{Careful with hardcore paraphrasing !!}

\subsection{Major points}


\todoin{To be completed}

\begin{itemize}
  \item Importance of dynamic models and learning from real-time data streams
  \item Approaches to online learning: static and dynamic
  \item Data streams
  \item Clustering time series
  \item Hoeffding bound
  \item Real-world problems and application solutions
\end{itemize}

\subsubsection{Discipline}

\todoin{To be completed}

This talk is obviously mainly relevant to Artificial Intelligence field, specifically Machine Learning algorithms. (Big) data analysis is the main subject of the talk, with an emphasize on real-time data streams.

\section{Air-Traffic Flow}

\section{BiDS Conference}


\bibliographystyle{alpha}
\bibliography{SEMINARS}

\end{document}
