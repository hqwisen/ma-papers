\documentclass[letterpaper]{article}
\usepackage{natbib}
\usepackage[utf8]{inputenc}
\usepackage[margin=3.5cm]{geometry}
\usepackage{listings}
\usepackage{adjustbox}
\usepackage{xcolor}
\usepackage{verbatim}
\usepackage{graphicx}%images
\usepackage{fancyhdr}%for headers and footers
\usepackage{adjustbox}
\usepackage{verbatim}
\usepackage{listings}
\usepackage{fancyhdr}
\usepackage{multirow}
\usepackage{amsmath}
\usepackage{array}
\usepackage{mathpazo}
\usepackage{subcaption}
\usepackage{float}
\usepackage{csvsimple}
\usepackage{filecontents}
\usepackage{lscape}
\usepackage{afterpage}
\usepackage{hyperref}
\usepackage{inconsolata}
\usepackage{color}


\begin{document}

\section{Cook-Levin Theorem: SAT is NP-complete}

To proof that we have to:

\begin{enumerate}
  \item SAT $\in$ NP
  \item any laguage in NP is $\leq_p$ to SAT
\end{enumerate}

Let $A$ a language in $NP$ and $N$
 be the NTM (Nondeterministric Turing Machine)
that decides $A$. The machine $N$ decides $A$ in $n^k$ time for some constant
$k$.

We will construct a tableau for $N$ on $w$ of size $n^k$ $\times$ $n^k$.
The tableau is constructed following these propreties:

\begin{itemize}
  \item Each row represent the configurations
  of a branch of computation of $N$ on $w$
  \item Each row starts and ends with the symbol $\#$
  \item The first row is the starting configuration and each row follows the
  previous according to $N$'s transition function
  \item A tableau is \textbf{accepting} if any row of the tableau
  is an accepting configuraton
\end{itemize}

The reduction build multiple tableau based on the computation tree, and
every accepting tableau corresponds to an accepting computation branch
of $N$ on $w$.

\subsection{Reduction}

On input $w$ the reduction produces a formula $\phi$.
$$
\phi = \phi_{cell} \wedge \phi_{start} \wedge \phi_{move} \wedge \phi_{accept}
$$

This formula is composed of a set of literals $x_{i, j, s}$ where
 $1 \le i, j \le n^k$ and  $s \in C$ with
 $C = Q \cup \cub \Gamma \cup \{\#\}$ (i.e.
 a symbol that can be in the tableau).

\subsubsection{Cell formula}

Ensure that each cell contains one and only one character:

$$
\phi_{cell} = \bigwedge_{1 \le i, j \le n^k}[\bigvee_{s \in C} x_{i, j, s}]
$$

The idea is the following: for each cell, at least one symbol must be true
and two symbols cannot be true together.

\subsubsection{Start formula}

Ensure that the first row is the start configuration of $N$ on $w$:

\subsubsection{Accept formula}

Guarantees that an accepting configurations occurs in the tableau

\subsubsection{Move formula}

Guarantees that each row of the tableau corresponds to a configuration
that legally follows the configuration of the preceding row according to
$N$'s rules. It does so by ensuring that each $2 \times 3$ window of the
tableau is \textbf{legal}.

A windows is legal when it respects the following transitions:
$\delta(q_1, a) = \{(q_1, b, R)\}$
and $\delta(q_1, b) = \{(q_2, c, L), (q_2, a, R)\}$



\end{document}
