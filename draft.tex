\documentclass[letterpaper]{article}
\usepackage{natbib}
\usepackage[utf8]{inputenc}
\usepackage[margin=3.5cm]{geometry}
\usepackage{listings}
\usepackage{adjustbox}
\usepackage{xcolor}
\usepackage{verbatim}
\usepackage{graphicx}%images
\usepackage{fancyhdr}%for headers and footers
\usepackage{adjustbox}
\usepackage{verbatim}
\usepackage{listings}
\usepackage{fancyhdr}
\usepackage{multirow}
\usepackage{amsmath}
\usepackage{array}
\usepackage{mathpazo}
\usepackage{subcaption}
\usepackage{float}
\usepackage{csvsimple}
\usepackage{filecontents}
\usepackage{lscape}
\usepackage{afterpage}
\usepackage{hyperref}
\usepackage{inconsolata}
\usepackage{color}


\begin{document}

\section{Cook-Levin Theorem: SAT is NP-complete}

(Answner from Sipser textbook) \\

The Cook-Levin Theorem: \textbf{SAT is NP-complete} \\

To proof that we have to:


\begin{enumerate}
  \item SAT $\in$ NP
  \item any laguage in NP is $\leq_p$ to SAT
\end{enumerate}

(1) A nondeterministic polynomial
time machine can guess an assignment to a given formula $\phi$
and accept if the
assignment satisfies $\phi$.

\paragraph{}


(2) Let $A$ a language in $NP$ and $N$
 be the NTM (Nondeterministric Turing Machine)
that decides $A$. The machine $N$ decides $A$ in $n^k$ time for some constant
$k$.

We will construct a tableau for $N$ on $w$ of size $n^k$ $\times$ $n^k$.
The tableau is constructed following these propreties:

\begin{itemize}
  \item Each row represent the configurations
  of a branch of computation of $N$ on $w$
  \item Each row starts and ends with the symbol $\#$
  \item The first row is the starting configuration and each row follows the
  previous according to $N$'s transition function
  \item A tableau is \textbf{accepting} if any row of the tableau
  is an accepting configuraton
\end{itemize}

The reduction build tableau based on the computation tree, and
every accepting tableau corresponds to an accepting computation branch
of $N$ on $w$.

\subsection{Reduction}

On input $w$ the reduction produces a formula $\phi$.
$$
\phi = \phi_{cell} \wedge \phi_{start} \wedge \phi_{move} \wedge \phi_{accept}
$$

This formula is composed of a set of literals $x_{i, j, s}$ where
 $1 \le i, j \le n^k$ and  $s \in C$ with
 $C = Q \cup \cup \Gamma \cup \{\#\}$ (i.e.
 a symbol that can be in the tableau).

\paragraph{$\phi_{cell}$}

Ensure that each cell contains one and only one character:

$$
\phi_{cell} = \bigwedge_{1 \le i, j \le n^k}\left[
\bigvee_{s \in C} x_{i, j, s}
\wedge
\left( \bigwedge_{s,t \in C, s \not= t}
\neg x_{i,j,s} \vee \neg x_{i,j,t}  \right)
\right]
$$

The idea is the following: for each cell, at least one symbol must be true
(first clause)
and two symbols cannot be true together (second clause) i.e. one symbol
can be true for a cell.

\paragraph{$\phi_{start}$}

Ensure that the first row is the start configuration of $N$ on $w$:

$$
\phi_{start} = x_{1, 1, \#} \wedge x_{1, 2, q_0}
\wedge x_{1, 3, w_1} \wedge x_{1, 4, w_2} \wedge ... \wedge x_{1, n+2, w_n}
\wedge x_{1, n+3, \text{\textvisiblespace}}
\wedge ... \wedge
\wedge x_{1, n^k-1, \text{\textvisiblespace}}
\wedge x_{1, n^k, \#}
$$

\paragraph{$\phi_{accept}$}

Guarantees that an accepting configurations occurs in the tableau:

$$
\bigvee_{1 \le i,j \le n^k} x_{i, j, q_{accept}}
$$

\paragraph{$\phi_{move}$}

Guarantees that each row of the tableau corresponds to a configuration
that legally follows the configuration of the preceding row according to
$N$'s rules. It does so by ensuring that each $2 \times 3$ window of the
tableau is \textbf{legal}.

A windows is legal when it respects the following transitions:
$\delta(q_1, a) = \{(q_1, b, R)\}$
and $\delta(q_1, b) = \{(q_2, c, L), (q_2, a, R)\}$.
With $a_1$, ..., $a_6$ a legal window, we have the following formula: \\

$$
\phi_{move} = \bigwedge_{i \le i < n^k, \;\; 1 < j < n^k}
\text{$(i, j)$-window is legal}
$$

$$
\bigvee_{a_1, .., a_6\text{  is a legal window}} = (x_{i, j-1, a_1} \wedge
x_{i, j, a_2} \wedge
x_{i, j+1, a_3} \wedge
x_{i+1, j-1, a_4} \wedge
x_{i+1, j, a_5} \wedge
x_{i+1, j+1, a_6})
$$

Next, we analyze the complexity of the reduction
by looking at the size of $\phi$. The tableau contains $n^{2k}$ cells.
Each cells has $l$ variables associated with $l$ being the size of $C$.
Because $l$ depends only on the TM $N$ and not on the input size $n$,
the total
number of variables in $O(n^{2k})$. $\phi_{cell}$,
$\phi_{accept}$ and $\phi_{move}$ contains a fixed size
formula for each cell \textit{i.e.} their size is $O(n^{2k})$.
$\phi_{start}$ has a fragment for
each cell in the top row, so its size is $O(n^k)$. Total size of $\phi$
is $O(n^{2k})$. This is sufficient and shows that the size of $\phi$ is
polynomial in $n$, thus the reduction can be done in polynomial time.

\end{document}
