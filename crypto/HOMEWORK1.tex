\documentclass[letterpaper]{article}
\usepackage{natbib}
\usepackage[utf8]{inputenc}



\title{Bitcoin fundamentals}
\author{Hakim Boulahya \\
Université Libre de Bruxelles\\
hboulahy@ulb.ac.be}


\begin{document}
\maketitle
% \begin{abstract}

% \end{abstract}


\section*{Introduction}

\paragraph{}
Bitcoin is a electronic cash system in which it is possible to exchange
money, in form of coins, anonymously.
The transaction are recorded
using cryptographic methodologies such as digital signatures,
public key encryption and hash algorithms.

\paragraph{}
It is necessary to have a structure that will verify payements to
avoid coins double-spending problem.
The major advancement of the Bitcoin protocol
is the decentralized transaction management structure,
using a peer-to-peer architecture and a proof-of-work system.

\section{Cryptographic basics}
Before describing the Bitcoin system, it is necessary to provide definitions
and explain basic cryptographic mechanisms used by Bitcoin.

\subsection{Digital signatures}

\paragraph{}
To define a transaction, it is necessary first to define a digitial signature.
A digitial signature is a mathematical scheme process that must ensure
the following three propreties: (1) the \textbf{integrity} of the signed document,
\textit{i.e.} the document cannot be modified during its transit. (2) Ensure
the \textbf{authenticity} \textit{i.e.} verifying the owner.
(3) It also provides the hability to prove the \textbf{non-repudiation}
of a document \textit{i.e.} ensure that all entities that signed the
document accept and confirm the document content.

\subsection{Public-key encryption}

\paragraph{}
A mechanism that provide the digital signature propreties is the public-key
encryption system \cite{RSA}. It is composed of a encryption algorithm $E$, a
decryption algorithm $D$ and two keys, one public key $k$ and one private key
$k'$. The public-key is shared to everyone, and the private-key is
kept secret by the owner.

\paragraph{}
A public encryption systems must have both algorithms
easy to compute, when using the keys. That means the decryption (encryption)
algorithm is a trap-door one-way function \cite{DiffieHellman}, it
is difficult to compute except if a trap-door \textit{i.e.}
the private (public) key $k'$ is known.
\paragraph{}

An encryption is using the algorithm $E$ with parameter $k$
to return a cihpertext $C$ from a message $M$:
\begin{equation}
    E_k(M) = C
\end{equation}
A decryption is using the algorithm $D$ with parameter $k'$
to return a message $M$ from a ciphertext $C$:
\begin{equation}
    D_{k'}(C) = D_{k'}(E_k(M)) =  M
\end{equation}

\paragraph{}
It is also possible to recover from a encrypted
(decrypted) message $M$ using the private (public) key and the
corresponding algorithm, formally:

\begin{equation}
    D_{k'}(E_k(M)) = M
\end{equation}
\begin{equation}
    \label{e(d)}
    E_{k}(D_{k'}(M)) = M
\end{equation}
The equation (\ref{e(d)}) provide the signature mechanism. Since only
the owner can provide an output using the decryption algorithm $D$ and
the private key $k'$, everyone can verify the signature using the public key.


\subsection{Hash}

\paragraph{}
A hash function is an algorithm that will convert an arbitrary
size message into a fixed size digest. The hash function
used by bitcoin is SHA-256: it produces a digest of size 256 bits.
The important proprety of cryptographic hashes is that the design is made
such that the creation of a hash is efficient but hard to recover the
original message. If $h$ is the hash function $x$ the message and $y$ the hash
we have:
\begin{equation}
    h(x) = y
\end{equation}
$h(x)$ is easy to compute but $h^{-1}(y)$ is hard.

\section{Bitcoins protocol}

\subsection{Address}
\paragraph{}
The privacy of the transfers is ensures by the Bitcoin addresses. An address
is 160 bits hash. Each address
is generated using a pair of ECDSA public/private key that identify an owner
during a transaction. The transaction ledger is public, so it's not encouraged
to reuse an address, because it will be possible to link the transactions
make by an address and build a profile of the owner of this address to identify
him.

\subsection{Transaction and coins}
\paragraph{}
A transaction is represented by a hash.
An electronic coin is represented by a
chain of digital signatures \cite{bitcoin}.
If Alice, represented by the public private key pair $(a, a^{'})$,
want to transfer a coin to Bob, represented by $(b, b^{'})$. Her coin
is represented by the chain $\{t_0, t_1, .., t_n \}$, which is ordered
and $n$ is finite, where
$t_n$ is the last transaction made for this coin. Each transaction
$t_i \forall i \le n$ is a hash. To perform the transaction
Alice must sign, using
here private key $a^{'}$,
a hash composed of the transaction $t_n$ and Bob's public key $b$.
This will add a new
transaction $t_{n+1}$ to the coin. We can verify using Alice's public key $a$
that Bob's is the new owner of the coin.
This is the method proposed to verify the ownership of a coin.

\subsection{Block chain}
\paragraph{}
A block is a file that contains multiple transactions,
recorded permanently \cite{bitcoinwiki}. A block contains also a block header,
a 256 bits hash, used to the following block to create generate its own hash.
The blockchain is a ledger of all
recorded blocks. The miners, the computers that do the (proof-of-)work to
approve the transactions, add a new blocks in the blockchain in a
chronological order. Those new block records,
once added to the Bitcoin blockchain, can never be removed or modified.



\subsection{Double-spending}
\paragraph{}
We know how to verify the ownership of a coin, but we cannot make sure
that a owner doesn't spend the same coin twice, or more.
To be able to find a solution to this problem, we need a \textit{central}
authority to acknowledge each transaction. Bitcoin uses a peer-to-peer network,
a distributed application meaning that the data are not stored in a
central server but in multiple computers that form the p2p network,
and a proof-of-work system called hashcash \cite{hashcash}. Using a p2p network
provides a decentralized authority to approve the transactions and the proof-of-work
provides a system that makes it hard to rebuild a block therefor hard to modify
an approved transaction.

\subsection{Proof-of-work}
\paragraph{}
Each blocks is minted using the hashcash cost-fuction algorithm.
A cost-function should be easy to verify but hard to compute, and the
computation difficulty is parameterisable \cite{hashcash}.
\paragraph{}
Let $b_n$ the hash
of the last block in the blockchain.
We want to create a new block $b_{n+1}$. Bitcoins uses hashcash as follow:
each block hash requires a number of 0 bits at the beginning. To build this
hash, concatenated the previous block hash and a nonce, a 32 bits numbers
that is incremented. This concatenation is hashed using hashcash-SHA-256
and the result
must be a new 256 bits hash starting with the required number of 0. This new
hash must be lower than the previous block hash: $b_{n+1} < b_n$.

\bibliographystyle{plain}
\bibliography{HOMEWORK1}

\end{document}

%% Citation to make
% RSA
% Satoshi Bitcoin original paper
% Diffie Hellman
% hashcash
% wiki
