\documentclass[letterpaper]{article}
\usepackage{natbib}
\usepackage[utf8]{inputenc}



\title{Bitcoin fundamentals}
\author{Hakim Boulahya \\
Université Libre de Bruxelles\\
hboulahy@ulb.ac.be}


\begin{document}
\maketitle
% \begin{abstract}

% \end{abstract}


\section*{Introduction}
\paragraph{}

Bitcoin is a electronic cash system in which it is possible to exchange
money, in form of coins, anonymously.
The transaction are recorded
using cryptographic methodologies such as digital signatures and
public key encryption.
\paragraph{}

It is necessary to have a structure that will verify payements to
avoid coins double-spending problem.
The major advancement of the Bitcoin protocol
is the decentralized transaction management structure,
using a peer-to-peer architecture.


\section*{Digital signatures}
\paragraph{}

To define a transaction, it is necessary first to define a digitial signature.
A digitial signature is a mathematical scheme process that must ensure
the following three propreties: (1) the \textbf{integrity} of the signed document,
\textit{i.e.} the document cannot be modified during its transit. (2) Ensure
the \textbf{authenticity} \textit{i.e.} verifying the owner.
(3) It also provides the hability to prove the \textbf{non-repudiation}
of a document \textit{i.e.} ensure that all entities that signed the
document accept and confirm the document content.

\section*{Public-key encryption}

\paragraph{}
A mechanism that provide the digital signature propreties is the public-key
encryption system \cite{RSA}. It is composed of a encryption algorithm $E$, a
decryption algorithm $D$ and two keys, one public key $k$ and one private key
$k'$. The public-key is shared to everyone, and the private-key is
kept secret by the owner.

\paragraph{}
A public encryption systems must have both algorithms
easy to compute, when using the keys. That means the decryption (encryption)
algorithm is a trap-door one-way function \cite{DiffieHellman}, it
is difficult to compute except if a trap-door \textit{i.e.}
the private (public) key $k'$ is known.
\paragraph{}

An encryption is using the algorithm $E$ with parameter $k$
to return a cihpertext $C$ from a message $M$:
\begin{equation}
    E_k(M) = C
\end{equation}
A decryption is using the algorithm $D$ with parameter $k'$
to return a message $M$ from a ciphertext $C$:
\begin{equation}
    D_{k'}(C) = D_{k'}(E_k(M)) =  M
\end{equation}

\paragraph{}
It is also possible to recover from a encrypted
(decrypted) message $M$ using the private (public) key and the
corresponding algorithm, formally:

\begin{equation}
    D_{k'}(E_k(M)) = M
\end{equation}
\begin{equation}
    \label{e(d)}
    E_{k}(D_{k'}(M)) = M
\end{equation}
The equation (\ref{e(d)}) provide the signature mechanism. Since only
the owner can provide an output using the decryption algorithm $D$ and
the private key $k'$, everyone can verify the signature using the public key.


\section*{Hash}

A hash function is an algorithm that will convert an arbitrary
size message into a fixed size digest. The hash function
used by bitcoin is SHA-256: it produces a digest of size 256 bits.

\section*{Transaction and coins}

A transaction is represented by a hash.
An electronic coin is represented by a
chain of digital signatures \cite{bitcoin}.
If Alice, represented by the public private key pair $(a, a^{'})$,
want to transfer a coin to Bob, represented by $(b, b^{'})$. Her coin
is represented by the chain $\{t_0, t_1, .., t_n \}$, which is ordered
and $n$ is finite, where
$t_n$ is the last transaction made for this coin. Each transaction
$t_i \forall i \le n$ is a hash. To perform the transaction
Alice must sign, using
here private key $a^{'}$,
a hash composed of the transaction $t_n$ and Bob's public key $b$.
This will add a new
transaction $t_{n+1}$ to the coin. We can verify using Alice's public key $a$
that Bob's is the new owner of the coin.
This is the method proposed to verify the ownership of a coin.

\section*{Double-spending problem}

We know how to verify the ownership of a coin, but we cannot make sure
that a owner doesn't spend the same coin twice, or more.
To be able to find a solution to this problem, we need a \textit{central}
authority to acknoledge each transaction. Bitcoin uses a peer-to-peer network,
a distributed application meaning that the data are not stored in a
central server but in multiple computers that form the p2p network,
and a proof-of-work system \cite{hashcash}. Using a p2p network
provides a decentralized authority to approve the transactions and the proof-of-work
provides a system that makes it hard to rebuild a block, hard to modify
an approved transaction.


\section{Block}

A block cointains multiple transaction


\section{Address}

Address is the public private key. Bad reuse because no anonymousity.

An address is the how an owner is represented during a transaction.
It is possible to reuse an address, but it is not recommanded because
it will be possible to read all transaction made with this address, this
a way that can be used to identify a bitcoin user.

Same question as for the transaction


\section{Peer-to-peer and blockchain}

\section{Timestamp}

\section{Proof-of-work}

\end{document}

%% Citation to make
% RSA
% Satoshi Bitcoin original paper
% Diffie Hellman
% hashcash
