\documentclass[letterpaper]{article}
\usepackage{natbib}
\usepackage[utf8]{inputenc}
\usepackage{graphicx}
\usepackage{color}
\usepackage{multirow}
\usepackage{amsmath}
\usepackage{array}
\usepackage{subcaption}
\usepackage{mathpazo}
\usepackage[a4paper]{geometry}
\usepackage{float}

\title{Computability and Complexity: $\not=$SAT}
\author{\Large Hakim Boulahya \\
Université Libre de Bruxelles
}

\begin{document}
\maketitle
% \tableofcontents
% \newpage

\section*{Proof Idea}

To proof that $\not=$ SAT is NP-complete we have to prove that a NP problem is
reducible in polynomial time to $\not=SAT$.

a ou b ou c

f, f, f is not valid because it doesnt satisfy

List:

* Look at the bookmarks
* Proove NP-Complete: Proove that it is 3SAT by adding more formulas
confirming the !=assign

\section*{Proof}

Hint proof

If we have a diffassign that satifsfy the formula, it means that in each
clause there is either two true with a false two false with a true. By
taking the negation it will jjust convert the 2t1f to a 1t2f and the 2f1t to
1f2t which also satisfy the formula.


To proof that it is NP-complete we have to proove that all NP problem are
reducible in P time to diffSAT. Since 3SAT is NP-complete, by reducing it
to diffSAT it will prove that diffSAT is NP-Complete.

Note about the hint: the idea is to convert each clause into diffSAT.
Since we known that if a diffassign sat a clause, it will also sat the neg
of the clause.

3SAT clause: c = (x y z)

The idea is if (x y z) is true, i.e. an assignment sat the formula,
we must find a 3cnf-formula that is true also
but never contains three true in any clauses.

Actually when converting we must fine a formula that if c is sat, then
there exist a diffassing that sat the new formula.

1 1 1
1 1 0
1 0 1
1 0 0
0 1 1
0 1 0
0 0 1
0 0 0


if c = T

(c x y) (nc z F)

\end{document}
