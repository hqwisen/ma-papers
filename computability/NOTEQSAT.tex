\documentclass[letterpaper]{article}
\usepackage{natbib}
\usepackage[utf8]{inputenc}
\usepackage{graphicx}
\usepackage{color}
\usepackage{multirow}
\usepackage{amsmath}
\usepackage{array}
\usepackage{subcaption}
\usepackage{mathpazo}
\usepackage[a4paper]{geometry}
\usepackage{float}
\usepackage{fancyhdr}
\usepackage{lastpage}

\title{INFOF408 - Computability and Complexity \\
 \Large Homework}
\author{\Large Hakim Boulahya \\
Université Libre de Bruxelles
}

\pagestyle{fancy}

\fancyhf{}
  \renewcommand{\headrulewidth}{0pt}
\fancyfoot[C]{\thepage/ \pageref{LastPage}}

\fancypagestyle{firststyle}
{
\fancyfoot[C]{\thepage/ \pageref{LastPage}}
}

\begin{document}

\maketitle
% \tableofcontents
% \newpage

\thispagestyle{firststyle}


\section{$\not=$SAT is NP-complete}


\subsection{Proof Idea}

\paragraph{}

$\not=$SAT is a 3cnf-formula and we know that 3SAT is NP-complete.
It is trivial that the best way to proof that  $\not=$SAT is NP-complete
is to find a way reduce in polynomial time 3SAT to $\not=$SAT. Therefore
the goal is to find a way to construct an equivalent $\not=$SAT formula
for any 3SAT formula.
The  $\not=$SAT formula must be satisfies iff the corresponding
3SAT formula is satistied.

We will proof that for any  $\not=$assignment that
satisfy the  $\not=$SAT formula, the negation of this assignment also satisfy
the formula. Therefore we have to find an  $\not=$assignment that satifsfy
the newly constructed  $\not=$SAT formula.

\subsection{Notations}


Let $\phi$ be a 3cnf formula:

$$
\phi = (a_1 \vee b_1 \vee c_1) \wedge (a_2 \vee b_2 \vee c_2)
\wedge ... \wedge
(a_{k-1} \vee b_{k-1} \vee c_{k-1}) \wedge (a_k \vee b_k \vee c_k)
$$



\subsection{Negation of any
$\not=$assignment is also a $\not=$assignment}
\label{neg}
\paragraph{}

If $\phi$ has a $\not=$assignment, then it means that there exists an
assignment such that $\phi$ is satified and there is at least one
literal for each clause that is set to false, and not all literals
of this clause are set false. By negating the assignment, the literal
set to false will change to true, which will still satisfy the clause.


\subsection{Proof}

Let $\phi$ represents any 3SAT formula. It
is satisfied if all clauses are satified. To construct a
$\not=$SAT formula, we will create, for each initial clause
$(a_i \vee b_i \vee c_i)$,
two clauses
and add a new literal $x_i$ such that:

$$
A_{i} = (a_i \vee b_i \vee x_i)
$$
$$
B_{i} = (c_i \vee \neg x_i \vee \bot)
$$

For each 3SAT formula, we have a $\not=$SAT formula:

$$
\phi_{\not=} = A_1 \wedge B_1 \wedge A_2 \wedge B_2 \wedge ...
\wedge A_{k-1} \wedge B_{k-1} \wedge A_k \wedge B_k
$$

To make sure that $\phi_{\not=}$ is a $\not=$SAT formula it is
necessary that: (1)
each constructed clauses
must be satisfied only if the initial clause is satisfied
(2) the new clauses are satifisied with a $\not=$assignment.

The construction is made as follow: if the assignment that satisfy the
initial clause is when $c_i = \top$ and either $a_i = \top$, $b_i = \top$ or
$a_i = \bot$ and $b_i = \bot$, \textit{i.e.}
$a_i$ and $b_i$ must not be true together when
$c_i$ is true,
then set $x_i = \top$. For all other assignments set $x_i = \bot$.
The construction is shown in Fig. \ref{tab:assignments}.

\begin{figure}[H]
    \centering
    \begin{tabular}{|c|c|c||c|}
        \hline
        $a_i$ & $b_i$ & $c_i$ & $x_i$ \\
        \hline
        \hline
        1 & 0 & 1 & 1 \\
        \hline
        0 & 1 & 1 & 1 \\
        \hline
        0 & 0 & 1 & 1 \\
        \hline
        1 & 1 & 1 & 0 \\
        \hline
        1 & 1 & 0 & 0 \\
        \hline
        1 & 0 & 0 & 0 \\
        \hline
        0 & 1 & 0 & 0 \\
        \hline
        0 & 0 & 0 & 0 \\
        \hline
    \end{tabular}

    \caption{Construction of $x_i$ based on the initial clause}
    \label{tab:assignments}
\end{figure}

First let's prove that the construction in Fig. \ref{tab:assignments}
satisfy the new clauses only
when the initial clause is satisfy. The initial clause is not sat when
all literals are set to false: $x_i$ is set to false in that case,
therefore $A_{i}$ is not sat \textit{i.e.} $\phi_{\not=}$
will not be sat. When the initial clause is sat, then the
new clauses are both sat: if the assignment that satisfy the initial clause
is based on $a_i$ or $b_i$ (\textit{i.e.} they are true), $A_i$ is true,
and $B_i$ is true because of the negation of $x_i$. And if the sat assignment
is only based on $c_i$, then $x_i$ is set to true, which sat $B_i$ and $A_i$.

\paragraph{}

Second we have to prove that that the construction in Fig. \ref{tab:assignments}
is a $\not=$assignment for $A_i \wedge B_i$.
It is obvious that $B_i$ is satisfied with a $\not=$assignment because of
the constant literal $\bot$.
We can see that any
assignment in Fig. \ref{tab:assignments} that satisfy $A_i$,
still satisfy $A_i$
when using their negation (see \ref{neg}).

\paragraph{}

This construction is obviously polynomial because we add a new clause per
initial clause from $\phi$, and we only assign values the the $x$ literal
based on constant complexity conditions.

\paragraph{}

We proved that $\quad 3SAT \le_{P} \not=SAT \quad$
\textit{i.e.} $\not=$SAT is NP-complete.

\end{document}
