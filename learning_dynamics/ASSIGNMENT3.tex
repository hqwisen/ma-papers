\documentclass[letterpaper]{article}
\usepackage{natbib}
\usepackage[utf8]{inputenc}
\usepackage{graphicx}
\usepackage{color}
\usepackage{multirow}
\usepackage{amsmath}
\usepackage{array}
\usepackage{subcaption}
\usepackage{mathpazo}
\usepackage[a4paper]{geometry}
\usepackage{float}


\title{
\LARGE Learning Dynamics: Assignment 3 \\
\Large Multi-Armed Bandits and Stochastic Reward Game}
\author{\Large Hakim Boulahya \\ \\
hboulahy@ulb.ac.be - 000391737 \\ \\
\large Université Libre de Bruxelles \\
}

\begin{document}
\maketitle
\tableofcontents
\newpage

\section{N-Armed Bandit}

\paragraph{Remark} About the notation, the first arm/action starts
at \#0.

\subsection{Exercice 1}

\subsubsection{Average rewards}


% ex1 rewards
\begin{figure}[H]
    \centering
    \includegraphics[width=.7\linewidth]{images/assign3/ex1/rewards}
    \caption{Average rewards for all algorithms}
    \label{fig:rewards_ex1}
\end{figure}

\paragraph{}

We can see that only exploring \textit{i.e.} random
and softmax with larger temperate $\tau = 1$ doesn't perform well, and
the reward stays low. Using softmax
with a temperature of 0.1 gives better results because
of the fact that it gets more greedy.

Using $\epsilon$-greedy with small
value, i.e. a large probability to choose the optimal actions,
tends
to better rewards. egreedy with 0 seems to arrives
\textit{directly} to the best arm. We can see on the graph that when $\epsilon$
is bigger it usually takes more time to get to the best arm, because of
the exploration.
Softmax with a temperature of 0.1 does tends to the best arm, but takes more
time.

\paragraph{}


It is important to note that the greedy behaviour is better, because
the standard deviation is small for this exercice. Because of that, using
the best action based on the estimated Q-values is accurate.

\subsubsection{Q-values estimation}

\label{sec:ex1_q_estim}

Figure \ref{fig:qtas_ex1} shows the estimation
of the Q-values for all algorithms.

% ex1 qtas
\begin{figure}[H]
  \begin{subfigure}{.5\textwidth}
    \centering
    \includegraphics[width=1\linewidth]{images/assign3/ex1/qta_0}
    \caption{$Q_{a_{0}}(t)$}
    \label{fig:qta_0_ex1}
  \end{subfigure}
  \begin{subfigure}{.5\textwidth}
    \centering
    \includegraphics[width=1\linewidth]{images/assign3/ex1/qta_1}
    \caption{$Q_{a_{1}}(t)$}
    \label{fig:qta_1_ex1}
  \end{subfigure}
  \begin{subfigure}{.5\textwidth}
    \centering
    \includegraphics[width=1\linewidth]{images/assign3/ex1/qta_2}
    \caption{$Q_{a_{2}}(t)$}
    \label{fig:qta_2_ex1}
  \end{subfigure}
  \begin{subfigure}{.5\textwidth}
    \centering
    \includegraphics[width=1\linewidth]{images/assign3/ex1/qta_3}
    \caption{$Q_{a_{3}}(t)$}
    \label{fig:qta_3_ex1}
  \end{subfigure}

    \caption{Plot per arm showing
    the $Q^{*}_{a_{i}}$
    of that action along with the actual $Q_{a_{i}}$ a i estimate over time
    with
    $\mu$ = (1.3, 1.1, 0.5, 0.3), $\sigma$ = (0.9, 0.6, 0.4, 2.0)}
    \label{fig:qtas_ex1}
\end{figure}



\subsection*{Random}

Since random methods is full exploration, we can see that for all Q-values
the estimation of the $Q^*_{a_i}$ are accurate.


\subsection*{$\epsilon$-greedy}

\paragraph{$\epsilon = 0$} Using a \textit{only} greedy exploration, we
can see that only the best actions have Q-values different from 0. Actually
the best action \#0 tends to be estimated fast enough. For action \#1, the
Q-values is bigger than 0 because at the beginning of the learning, the
algorithm is still looking for the greedy action, therefore action \#1
could be the best one, until the algorithm find that \#action 0 is better.
The two other actions, are never explored since the it is a full greedy
method.

\paragraph{$\epsilon = \{0.1, 0.2\}$}

The estimation is more accurate when $\epsilon > 0$ \textit{i.e.} when
the action are also selected randomly which allows exploration. We can see
that by using a non-null $\epsilon$, the estimation of the Q-values overtime
tends to be accurate. An intersting observation is that the estimation
of the Q-values is faster (the accuration rate)
when $\epsilon$ is larger. This is logical,
since more $\epsilon$ is large, more the action are selected randomly.

\subsection*{Softmax}

\paragraph{$\tau = 1$} Using $\tau = 1$ (a large enough temperature)
the Q-values estimation follows the same pattern as for any exploration methods
like random or $\epsilon = 0$. We can see that except for the last action
the curves are close to the random ones,
which confirm the exploration nature of a large temperature.
The fact that the curves of
the last action \#3 are more biased is because the standard deviation
is larger that for the other actions, which make the Q-values estimation
harder to make.


\paragraph{$\tau = 0.1$} Using a smaller temperature for the softmax
selection method, we have a more greedy action selection.
This action selection method is the worst one that estimate Q-values
(excluding the full exploiting method 0-greedy). We can see that
for the best arms \#0 and \#1, the estimations are larger than for there
worst arms, but it is \textit{far} from the actual $Q^*$. This can be explained
by the nature of the softmax selection method with a
small $\tau$: it sticks to the best arm
found and doesn't explore much to estimate the correct Q-values overtime.

\subsubsection{Histograms}

\label{sec:ex1_histo}

Figure \ref{fig:arms_ex1} shows the histograms for all algorithms.

% EX1 arms histograms
\begin{figure}[H]
  \begin{subfigure}{.5\textwidth}
    \centering
    \includegraphics[width=1\linewidth]{images/assign3/ex1/arms_random}
    \caption{}
    \label{fig:arms_random_ex1}
  \end{subfigure}
  \begin{subfigure}{.5\textwidth}
    \centering
    \includegraphics[width=1\linewidth]{images/assign3/ex1/arms_e_greedy0}
    \caption{}
    \label{fig:arms_e_greedy0_ex1}
  \end{subfigure}
  \begin{subfigure}{.5\textwidth}
    \centering
    \includegraphics[width=1\linewidth]{images/assign3/ex1/arms_e_greedy01}
    \caption{}
    \label{fig:arms_e_greedy01_ex1}
  \end{subfigure}
  \begin{subfigure}{.5\textwidth}
    \centering
    \includegraphics[width=1\linewidth]{images/assign3/ex1/arms_e_greedy02}
    \caption{}
    \label{fig:arms_e_greedy02_ex1}
  \end{subfigure}
  \begin{subfigure}{.5\textwidth}
    \centering
    \includegraphics[width=1\linewidth]{images/assign3/ex1/arms_softmax1}
    \caption{}
    \label{fig:arms_softmax1_ex1}
  \end{subfigure}
  \begin{subfigure}{.5\textwidth}
    \centering
    \includegraphics[width=1\linewidth]{images/assign3/ex1/arms_softmax01}
    \caption{}
    \label{fig:arms_softmax01_ex1}
  \end{subfigure}
    \caption{Histograms showing the number of times each action is selected
    per selection strategy with
    $\mu$ = (1.3, 1.1, 0.5, 0.3), $\sigma$ = (0.9, 0.6, 0.4, 2.0)}
    \label{fig:arms_ex1}
\end{figure}

\subsection*{Random}

\paragraph{}

The action selections, shown in Figure \ref{fig:arms_random_ex1},
for the random selection method is equidistributed.
Since all actions have the same probablity it is logical that the selection
is equidistributed when there $t \to \infty$.

\subsection*{$\epsilon$-greedy}


Figures \ref{fig:arms_e_greedy0_ex1}
\ref{fig:arms_e_greedy01_ex1}
\ref{fig:arms_e_greedy02_ex1} shows the histograms of action selections
for $\epsilon$-greedy selection methods.

\paragraph{$\epsilon = 0$}

With a null $\epsilon$, the action selection will be greedy \textit{i.e.},
and since we have a pretty small standard deviation, the best action \#0
is always chosen when found by the algorithm. This means that there is
no exploration, the method chooses the best arm directly. The fact that
the other actions are chosen, is that at the beginning it is necessary
to explore a little bit to find the best action.

\paragraph{$\epsilon = \{0.1, 0.2\}$}

It follows the same pattern that when $\epsilon$ is null, except that here
it is possible for the methods to explore more \textit{i.e.} choosing
other actions than the best one. This is why the bar of other actions
are higher when $\epsilon$ is bigger. If $\epsilon = 1$ the bars will
be as shown for the random methods in Figure \ref{fig:arms_random_ex1}.

\subsection*{Softmax}


Figures \ref{fig:arms_softmax1_ex1} \ref{fig:arms_softmax01_ex1}
shows the histograms of action selections
for softmax selection methods.

When temperature $\tau = 1$, the action selections are more randomize
\textit{i.e.} the exploration is more noticable. An interesting observation
in Figure \ref{fig:arms_softmax01_ex1}
is that when $\tau = 0.1$, \textit{i.e.}
the action selection is more greedy, softmax
algorithm tends to hesitate to choose the best actions. The
number of times that the 2 best actions,
action \#0 et \#1, are chosen is more or less equal.

\subsubsection{Questions}

\paragraph{
Which algorithms arrive close to the best arm ?
}

Based on the analysis that we made on the plots from this section,
we observed that the algorithms that exploit are getting closer to
the best arm. The \textit{full} exploit algorithm 0-greedy is the
fastest one followed by 0.1-greedy and 0.2-greedy. Using a low
temperature $\tau = 0.1$ with the softmax algorithm also allow exploitation,
but we can see that it is slower that the $\epsilon$-greedy ones.


\subsection{Exercice 2}

\subsubsection{Average rewards}

% ex2 rewards
\begin{figure}[H]
    \centering
    \includegraphics[width=.7\linewidth]{images/assign3/ex2/rewards}
    \caption{Average rewards for all algorithms}
    \label{fig:rewards_ex2}
\end{figure}


\paragraph{}

The results order of the algorithms that perform best doesn't really changes
compared to exercice 1.
By doubling the standard deviation, we only allowed the algorithms to
provide a more random rewards \textit{i.e.} a bigger range. We can see
that the dynamics of the algorithms don't change.

\paragraph{}

It is harder to learn because of the standard deviation that provides
a more randomize outcomes, therefore harder to find the best rewards.


\subsubsection{Q-values estimation}


Figure \ref{fig:qtas_ex2}
shows the estimation of the Q-values for all algorithms.

% ex2 qtas
\begin{figure}[H]
  \begin{subfigure}{.5\textwidth}
    \centering
    \includegraphics[width=1\linewidth]{images/assign3/ex2/qta_0}
    \caption{$Q_{a_{0}}(t)$}
    \label{fig:qta_0_ex2}
  \end{subfigure}
  \begin{subfigure}{.5\textwidth}
    \centering
    \includegraphics[width=1\linewidth]{images/assign3/ex2/qta_1}
    \caption{$Q_{a_{1}}(t)$}
    \label{fig:qta_1_ex2}
  \end{subfigure}
  \begin{subfigure}{.5\textwidth}
    \centering
    \includegraphics[width=1\linewidth]{images/assign3/ex2/qta_2}
    \caption{$Q_{a_{2}}(t)$}
    \label{fig:qta_2_ex2}
  \end{subfigure}
  \begin{subfigure}{.5\textwidth}
    \centering
    \includegraphics[width=1\linewidth]{images/assign3/ex2/qta_3}
    \caption{$Q_{a_{3}}(t)$}
    \label{fig:qta_3_ex2}
  \end{subfigure}

    \caption{Plot per arm showing
    the $Q^{*}_{a_{i}}$
    of that action along with the actual $Q_{a_{i}}$ a i estimate over time
    with
    $\mu$ = (1.3, 1.1, 0.5, 0.3), $\sigma$ = (1.8, 1.2, 0.8, 4.0)}
    \label{fig:qtas_ex2}
\end{figure}

\paragraph{}

The behaviour of the estimations for all algorithms are the same
that we explain in section \ref{sec:ex1_q_estim}. The major difference
is that the Q-values estimations for all algorithms, when exploring
takes more time, because the standard deviation is larger. Because of
that it is harder for the learner to estimate the correct Q-values.


\subsubsection{Histograms}


% ex2 arms histograms
\begin{figure}[H]
  \begin{subfigure}{.5\textwidth}
    \centering
    \includegraphics[width=1\linewidth]{images/assign3/ex2/arms_random}
    \caption{}
    \label{fig:arms_random_ex2}
  \end{subfigure}
  \begin{subfigure}{.5\textwidth}
    \centering
    \includegraphics[width=1\linewidth]{images/assign3/ex2/arms_e_greedy0}
    \caption{}
    \label{fig:arms_e_greedy0_ex2}
  \end{subfigure}
  \begin{subfigure}{.5\textwidth}
    \centering
    \includegraphics[width=1\linewidth]{images/assign3/ex2/arms_e_greedy01}
    \caption{}
    \label{fig:arms_e_greedy01_ex2}
  \end{subfigure}
  \begin{subfigure}{.5\textwidth}
    \centering
    \includegraphics[width=1\linewidth]{images/assign3/ex2/arms_e_greedy02}
    \caption{}
    \label{fig:arms_e_greedy02_ex2}
  \end{subfigure}
  \begin{subfigure}{.5\textwidth}
    \centering
    \includegraphics[width=1\linewidth]{images/assign3/ex2/arms_softmax1}
    \caption{}
    \label{fig:arms_softmax1_ex2}
  \end{subfigure}
  \begin{subfigure}{.5\textwidth}
    \centering
    \includegraphics[width=1\linewidth]{images/assign3/ex2/arms_softmax01}
    \caption{}
    \label{fig:arms_softmax01_ex2}
  \end{subfigure}
    \caption{Histograms showing the number of times each action is selected
    per selection strategy with
    $\mu$ = (1.3, 1.1, 0.5, 0.3), $\sigma$ = (1.8, 1.2, 0.8, 4.0)}
    \label{fig:arms_ex2}
\end{figure}

\paragraph{}

The actions selection behaviour for all algorithms seems to follow
the same pattern that we explain in section \ref{sec:ex1_histo}.
The major difference is that for the one that exploit th most, for example
$0$-greedy, the selection of the \textit{worst} action have a higher
bars. This is because of the larger standard deviation for all actions.
It is harder to determine which one is the optimal selection \textit{i.e.}
the exploration at the beginning is necessary, because the Q-values
are estimated delayed compared to when the standard deviation is smaller.

\subsubsection{Questions}

\paragraph{Harder or easier that Exercice 1 ?}

It is harder. Due to the higher standard deviation, the Q-values are
harder to estimate, therefore hard to tend to the best arm.

\paragraph{What about performance ?}

The performance changes are noticable.
By that we mean that the Q-values estimation
are more inaccurate because of the scale of the randomness that is larger.
It takes more time to estimate correctly, and the rewards curves are less
accurate \textit{i.e.} they tends to the same range of values. As we can
see in Figure \ref{fig:rewards_ex2}, the reward lines interlace more
and the exploit algorithms tends to the same range of reward values. The
deviation is small enough to emphasize the fact that exploration algorithms
(random and softmax 1)
are less efficient.

\subsection{Exercice 3}

\subsubsection{Average rewards}

% ex3 rewards
\begin{figure}[H]
    \centering
    \includegraphics[width=.7\linewidth]{images/assign3/ex3/rewards}
    \caption{Average rewards for all algorithms}
    \label{fig:rewards_ex3}
\end{figure}

\paragraph{Dynamic $\epsilon$-greedy}

The curve of the dynamic egreedy follows the same behaviour than the static
ones. It grows fast to reach a steady state around the best arm. This
is due to the fact that $1/{\sqrt{t}}$ will have a high probability to
explore at the beginning, but will fall fast enough to be greedy overtime
(at around epoch 100).

\paragraph{Dynamic softmax}

The curve of the dynamic softmax is different from the static ones. Using
the temperature $4*\frac{1000 - t}{1000}$ makes the algorithm to choose
the action randomly at the beginning and during a \textit{long} moment.
It's only at around epoch 800
that the action selection will be more greedy, therefore
tends to the best arm. This is explained by the fact that the temperature
at $t = 0$ (when $t$ is small) the temperature is large e.g.:
$$
\tau = 4*\frac{1000 - t}{1000} = 4
$$

And when $t = 1000$ (when $t$ is large)
the temperature is low \textit{i.e.} the action selection
method is more greedy:

$$
\tau = 4*\frac{1000 - 1000}{1000} = 0
$$



\subsubsection{Q-values estimation}

\paragraph{}

Figure \ref{fig:qtas_ex3} shows the Q-values estimation overtime of
the dynamic selection methods $\epsilon$-greedy and softmax.

% ex3 qtas
\begin{figure}[H]
  \begin{subfigure}{.5\textwidth}
    \centering
    \includegraphics[width=1\linewidth]{images/assign3/ex3/qta_0}
    \caption{$Q_{a_{0}}(t)$}
    \label{fig:qta_0_ex3}
  \end{subfigure}
  \begin{subfigure}{.5\textwidth}
    \centering
    \includegraphics[width=1\linewidth]{images/assign3/ex3/qta_1}
    \caption{$Q_{a_{1}}(t)$}
    \label{fig:qta_1_ex3}
  \end{subfigure}
  \begin{subfigure}{.5\textwidth}
    \centering
    \includegraphics[width=1\linewidth]{images/assign3/ex3/qta_2}
    \caption{$Q_{a_{2}}(t)$}
    \label{fig:qta_2_ex3}
  \end{subfigure}
  \begin{subfigure}{.5\textwidth}
    \centering
    \includegraphics[width=1\linewidth]{images/assign3/ex3/qta_3}
    \caption{$Q_{a_{3}}(t)$}
    \label{fig:qta_3_ex3}
  \end{subfigure}

    \caption{Plot per arm showing
    the $Q^{*}_{a_{i}}$
    of that action along with the actual $Q_{a_{i}}$ a i estimate over time}
    \label{fig:qtas_ex3}
\end{figure}

\paragraph{Dynamic $\epsilon$-greedy}

Since the dynamic e-greedy method is randomize at the beginning and
more greedy overtime, the Q-values are correctly estimated and tends to
the correct $Q^*$ overtime.

\paragraph{Dynamic softmax}

We have a large temperature at the beginning and a smaller temperature overtime.
This leads to a algorithm that explore a lot at the beginning, therefore
estimate the Q-values very fast.

\subsubsection{Histograms}

% ex3 arms histograms
\begin{figure}[H]
    \begin{subfigure}{.5\textwidth}
    \centering
    \includegraphics[width=1\linewidth]{images/assign3/ex3/arms_e_greedy_t}
    \caption{}
    \label{fig:arms_e_greedy_t_ex3}
    \end{subfigure}
    \begin{subfigure}{.5\textwidth}
      \centering
      \includegraphics[width=1\linewidth]{images/assign3/ex3/arms_softmax_t}
      \caption{}
      \label{fig:arms_softmax_t_ex3}
    \end{subfigure}
    \caption{Histograms showing the number of times each action is selected
    per selection strategy}
    \label{fig:arms_ex3}
\end{figure}

\paragraph{Dynamic $\epsilon$-greedy}

The action selection tends to the optimal action \#0 very fast. This
is because that the probability $\epsilon$ drops very fast overtime, which
lead to the greedy selection behaviour.

\paragraph{Dynamic softmax}

Since dynamic softmax has a large temperature when $t$ is small, therefore
explore a lot, we can see that all actions are selected. Overtime
to selection will be more greedy, and since the Q-values are accurate
because of the exploration, when $t$ is high enough, the action \#0 and \#1
will be selected more often. That explains the \textit{stairs} form
of the histogram.

\subsubsection{Questions}

\paragraph{Fixed parameters vs dynamic parameters}

Dynamic $\epsilon$-greedy
doesn't seems to perform better that the static ones.
The resulting rewards tends to be identical. Observing the curve
in Figure \ref{fig:rewards_ex3} shows
that it takes more time to stabilize to the best arm.

Dynamic softmax seems to be better, if you consider the long term reward.
It takes to around epoch
1000 for the algorithm to find the best reward value in
comparaison to the softmax using a a temperature of 0.1, where it
is close to the best arm at around epoch 200. But not as close as
the dynamic softmax, where it is closer to the best Q-value, when it
is stabilized.


\section{Stochastic Reward Game}

\subsection{Algorithms}

\paragraph{}


I chose to implement the softmax action selection with a temperature
$\tau = 0.1$. For the heuristic algorithm, the FMQ heuristic from
[1]. The expected value function is set as follow:

$$
EV(a) = Q(a) + c * maxR(a)
$$

The only difference with the article is that we don't use use frequence of
the max reward. This is due to the fact that the reward are stochastic,
\textit{i.e.} each reward is random and will occur only once. $c$ is the
weight of the FMQ (importance of the heuristic compared
to the Q-values), and is set to 1.

The temperature of the boltzmann distribution is:

$$
\tau(t) = e^{-t} * \tau_{max} + \tau_{min}
$$

$\tau_{max}$ is the value that the temperature has at the first epoch, and
$\tau_{min}$ is the lowest temperature that can be reach overtime.

And the variables are set as follow:

\begin{itemize}
  \item $\tau_{max} = 10$
  \item $\tau_{min} = 0.001$
\end{itemize}

The FMQ heuristic consists of choosing the action based on a boltzmann
distribution \textit{i.e.} a softmax action selection method with
a decreasing temperature $\tau(t)$ and a expected values that is not only the
Q-values but the function $EV(a)$.

\paragraph{}

The Q-values are computed as in the N Bandits Problems, since the
learners should choose their action independently. For this game the difference
is with the rewards. With 3 actions per player and joint learning \textit{i.e.}
the reward is the same for both player and based on their actions,
their is 9 outcomes possible in opposition to independent learning where
the outcome is based only on the player actions, therefore 3 outcomes
possible.

\subsection{Results}

\paragraph{}

Figure \ref{fig:climbing_rewards} shows the three graphs of the
total collected reward per episode for all three cases.
20 iterations of 5000 time steps per simulation were ran, and the resulting
plots shows the average reward of all iterations combined.

\begin{figure}[H]
  \begin{subfigure}{.5\textwidth}
    \centering
    \includegraphics[width=1\linewidth]{images/assign3/exa/rewards}
    \caption{$\sigma_0 = \sigma_1 = \sigma = 0.2$}
    \label{fig:climbing_rewards_exa}
  \end{subfigure}
  \begin{subfigure}{.5\textwidth}
    \centering
    \includegraphics[width=1\linewidth]{images/assign3/exb/rewards}
    \caption{$\sigma_0 = 4$ and $\sigma_1 = \sigma = 0.1$}
    \label{fig:climbing_rewards_exb}
  \end{subfigure}
  \begin{subfigure}{.5\textwidth}
    \centering
    \includegraphics[width=1\linewidth]{images/assign3/exc/rewards}
    \caption{$\sigma_1 = 4$ and $\sigma_0 = \sigma = 0.1$}
    \label{fig:climbing_rewards_exc}
  \end{subfigure}
    \caption{Total collected reward per episode
    versus the episode number for softmax 0.1 and FMQ heuristic}
    \label{fig:climbing_rewards}
\end{figure}


By settings all standard deviation to $0.2$, the results plot is shown
in Figure \ref{fig:climbing_rewards_exa}. We can see that the FMQ heuristic
provide a better rewards. Actually the rewards are an average of the two
best joint actions $<a_1, b_1>$ and $<a_2, b_2>$. It seems that the
algorithm stick to one best joint-actions.

\paragraph{}

By setting $\sigma_0 =$ 4 \textit{i.e.} the best joint-action  $<a_1, b_1>$
Q-values will be harder to estimate. We can see in Figure
\ref{fig:climbing_rewards_exb} that the softmax method will choose one
equilibrium and stick with it. The FMQ heuristic method average rewards
shows that the method explore much more during each epoch, and provide a
more randomize outcomes.

\paragraph{}

By setting $\sigma_1 =$ 4 \textit{i.e.} the joint-action $<a_2, b_2>$
Q-values wil be harder to estimate. Since the standard deviation is high
for the second optimum, the algorithms sticks to the one they found.



\subsection{Discussion}

\paragraph{How will the learning process change
if we make the agents independent learners?}

If the agents learn independently, the rewards will be based only on their
own actions, and not on the joint-action. This yield to the process of
using only 3 possible values instead of 9. Therefore it will be harder
to estimate the Q-values and find the best action. the IL
will be slower at determining  the
optimal equilbrium. Also if the standard deviation are larger, the probabilty
of finding the best actions will be harder for IL than for JAL.

\paragraph{How will the learning process change if we make
the agents always select the action that according
to them will yield the highest reward (assuming the
other agent plays the best response)?}


This is the action selection made by the 0-greedy method. Since we are choosing
our best action when the other player chooses his best action, this provide
a \textit{full} greedy action selection. Because of that the optimum might
be harder to find because no exploration is made by the agents, and this
will lead to a big possiblity to miss the
optimum equilibrium.

\section{Project source files}

The project wa implemented in Python3. It uses a combination of
scripts and config files to implement the exercices.

There is two scripts that are used, with config files, to run the
simulation:

\begin{itemize}
  \item \textbf{nbandits.py}: scripts used for the first part
  \item \textbf{climbing.py}: scripts used for the second part
\end{itemize}

Config files for exercices of the first part are in the directory
\textbf{bandits\_exercices/}.

Config files for exercices of the second part are in the directory
\textbf{climbing\_exercices/}.

The config files are self-explanatory.
\subsection{How to run}

Install the dependencies (numpy and matplotlib): \\ \\
\texttt{pip install -r requirements.txt}
\\ \\
Run the scripts: \\ \\
\texttt{python3 \{nbandits.py, climbing.py\} config.py}
\\ \\
Example: \\ \\
\texttt{python3 nbandits.py bandits\_exercices/ex1\_table3.py}

\section*{References}

[1] Kapetanakis, S. and Kudenko, D. (2005).
\textit{Reinforcement
learning of coordination in heterogeneous cooperative multi-agent systems.}
Lecture Notes in
Computer Science
(including subseries Lecture Notes in Artificial Intelligence and Lecture
Notes in Bioinformatics), 3394 LNAI:119–131.

\end{document}
