\documentclass[letterpaper]{article}
\usepackage{natbib}
\usepackage[utf8]{inputenc}
\usepackage{graphicx}


\title{Learning Dynamics: Assignement 1}
\author{Hakim Boulahya \\
Université Libre de Bruxelles\\
hboulahy@ulb.ac.be}

\begin{document}
\maketitle
% \begin{abstract}

% \end{abstract}

\section{Sequential truel}

\paragraph{}

The diagram representing the subgames are drawn as trees in
Figure \ref{fig:t1}, \ref{fig:t2} and \ref{fig:t3}. When we will refer to $T_1$,
$T_2$ and $T_3$ it means that we are refering to, respectively,
Figure \ref{fig:t1}, \ref{fig:t2} and \ref{fig:t3}. In the subgames,
the action $t(i)$ where $i \in \{A, B, C\}$, means that the current player
is targeting player $i$. The current player can of course not targeting himself.

\paragraph{}

The main game is represented by
$T_1$. As mentioned in the
assignement specifications, the subgames when $A$ fails to hits in intended
target are the same,  it is $T_2$. When in subgame $T_2$, we also found
that when $B$ misses is intended target, the subgames are the same, it is $T_3$.

\subsection{Preferences}
\begin{itemize}
    \item Players prefer outcomes with fewer people
    \item Players prefer to stay alive
\end{itemize}


\subsection{Subgame perfect equilibria}

\paragraph{C equilibrium in $T_3$}

To find the SPE, we have to use the backward induction, so we need to find
the SPE of $T_3$ first. Player $C$ always stay alive whatever target she is
choosing. The SPE is $t(A)$ if $p_a > p_b$, and $t(B)$ if $p_b > p_a$.

\paragraph{B equilibrium in $T_2$}

Let's consider subgame $T_2$. Player $B$ needs either to target $A$
or target $C$. If $B$ targets $A$, she has less chance to survive because,
since next turn $C$ will play and will still be alive whatever the result is
of this shoot, so it $B$ will always target $C$ in $T_3$.
Formally, if $B$ misses her outcome will be the same, because it will have
the SPE of subgame $T_3$, which is unique.
If $B$ targets $A$ and hits her, she has
a probabilty of $1 - p_c$ to stay alive. If $B$ targets $C$ and hits her, she
has a probablity of $1$ to stay alive. So $B$ will always choose to target $C$.

\paragraph{A equilibrium in $T_1$}

Let's consider the full game, subgame $T_1$. Intuitively, it is best for
$A$ to target the player with the biggest probability, because she will have
more chance to stay alive if she manage to eliminate the strongest opponent.
Formally, if $A$ misses her intended target, the outcome will be the same
since $T_2$ has an unique SPE. If $A$ targets $B$ and hits her, she has
a probability of $1 - p_c$ to stay alive, because $C$ is
the remaining shooter and has a probability of $1 - p_c$ to fail.
If $A$ targets $C$ and hits her,
she has a probability of $1 - p_b$ to stay alive, because $B$ is
the remaining shooter and has a probability of $1 - p_b$ to fail.
So $A$ targets $B$ if
$ (1 - p_c) > (1 - p_b)$, which can be simplify as
$p_c < p_b$, and $A$ targets $C$
if $(1 - p_b) > (1 - p_c)$, which can be simplify as $p_b < p_c$.


\paragraph{Weakness is strength}

In the previous paragraph, we explained that if $p_c > p_b$
$A$ will target $C$. If $C$ is the target, the her probablity of survival
is the probability that $A$ misses and $B$ misses,
formally
\begin{equation}
    (1 - p_a)(1 - p_b) = 1 - p_b - p_a + p_ap_b = 1 - p_a - p_b(1 - p_a)
\end{equation}

\begin{figure}[h!]
 \centerline{\includegraphics[scale=0.5]{images/T1}}
 \caption{$T_1$, main subgame}
 \label{fig:t1}
\end{figure}

\begin{figure}[h!]
 \centerline{\includegraphics[scale=0.5]{images/T2}}
 \caption{$T_2$, subgame when $A$ misses her intended target}
 \label{fig:t2}
\end{figure}

\begin{figure}[h!]
 \centerline{\includegraphics[scale=0.5]{images/T3}}
 \caption{$T_3$, subgame when $A$ and $B$ miss their intended targets}
 \label{fig:t3}
\end{figure}

\end{document}
