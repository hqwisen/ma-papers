\documentclass[letterpaper]{article}
\usepackage{natbib}
\usepackage[utf8]{inputenc}
\usepackage[margin=3.5cm]{geometry}
\usepackage{listings}
\usepackage{adjustbox}
\usepackage{xcolor}
\usepackage{verbatim}
\usepackage{graphicx}%images
\usepackage{fancyhdr}%for headers and footers
\usepackage{adjustbox}
\usepackage{verbatim}
\usepackage{listings}
\usepackage{fancyhdr}
\usepackage{multirow}
\usepackage{amsmath}
\usepackage{array}
\usepackage{mathpazo}
\usepackage{subcaption}
\usepackage{float}
\usepackage{csvsimple}
\usepackage{filecontents}
\usepackage{lscape}
\usepackage{afterpage}
\usepackage{hyperref}
\usepackage{inconsolata}
\usepackage{color}

\definecolor{pblue}{rgb}{0.13,0.13,1}
\definecolor{pgreen}{rgb}{0,0.5,0}
\definecolor{pred}{rgb}{0.9,0,0}
\definecolor{pgrey}{rgb}{0.46,0.45,0.48}

\lstset{language=Java,
  showspaces=false,
  showtabs=false,
  breaklines=true,
  showstringspaces=false,
  breakatwhitespace=true,
  commentstyle=\color{pgreen},
  keywordstyle=\color{pblue},
  stringstyle=\color{pred},
  basicstyle=\ttfamily,
  moredelim=[il][\textcolor{pgrey}]{$ $},
  moredelim=[is][\textcolor{pgrey}]{\%\%}{\%\%}
}



\hypersetup{
    colorlinks,
    citecolor=black,
    filecolor=black,
    linkcolor=black,
    urlcolor=black
}


% ------ HEADERS AND FOOTERS -----------
% \lhead{INFO-F403}
% \rhead{Project Report - Part 1}
%\pagestyle{fancy}
% \rfoot{\thepage}
%\cfoot{}
%\lfoot{Academic Year 2017-18}

\newcommand{\HRule}{\rule{\linewidth}{0.5mm}} %newcommand for cover page

\begin{document}

\begin{titlepage}
\begin{center}


\textsc{\LARGE universit\'e libre de bruxelles}\\[1.0cm]
\textsc{\Large D\'epartment d'Informatique}\\[1.5cm]

% Upper part of the page. The '~' is needed because \\
% only works if a paragraph has started.
% \includegraphics[width=0.3\textwidth]{image/ulblogo.jpg}~\\[1cm]

\textsc{
\large INFO-F404 \\
\Large  Real-Time Operating Systems
 \\[1cm]}
% Title
\HRule \\[0.7cm]

{ \huge \bfseries Project 1 – Audsley  \\[0.7cm] }

\HRule \\[2cm]

% Author and supervisor
\noindent
\begin{center} \large

\emph{Author:}\\
\Large Hakim \textsc{Boulahya}\\
Youcef \textsc{Bouharaoua}
\end{center}
\begin{center} \large

% \emph{Professor:} \\
% Gilles \textsc{Geeraerts}\\

\end{center}

\vfill

% Bottom of the page
{\large \today}

\end{center}
\end{titlepage}

\tableofcontents
\newpage

\section{Simulator}

\paragraph{Implementation choices}

The simulator of the single FTP simulator is implemented in a discret manner
from \texttt{start} to \texttt{stop}. For each time steps
we execute functions that will update the
current running job informations, the different requests (arrivals or deadlines)
and will store them in datasets. We choose not to print the different
actions during the simulation,
since the simulation is used for other commands such as
the plotter or the audsley algorithm. The output can be produced by
a call to the method \texttt{FTPSimulation.output()}.

\paragraph{}




- necessary to run the full simulation to be sure that it works
- Using the hyper period vs calculation with modulo


\end{document}
