\documentclass[letterpaper]{article}
\usepackage{natbib}
\usepackage[utf8]{inputenc}
\usepackage[margin=3.5cm]{geometry}
\usepackage{listings}
\usepackage{adjustbox}
\usepackage{xcolor}
\usepackage{verbatim}
\usepackage{graphicx}%images
\usepackage{fancyhdr}%for headers and footers
\usepackage{adjustbox}
\usepackage{verbatim}
\usepackage{listings}
\usepackage{fancyhdr}
\usepackage{multirow}
\usepackage{amsmath}
\usepackage{array}
\usepackage{mathpazo}
\usepackage{subcaption}
\usepackage{float}
\usepackage{csvsimple}
\usepackage{filecontents}
\usepackage{lscape}
\usepackage{afterpage}
\usepackage{hyperref}
\usepackage{inconsolata}
\usepackage{color}


\definecolor{pblue}{rgb}{0.13,0.13,1}
\definecolor{pgreen}{rgb}{0,0.5,0}
\definecolor{pred}{rgb}{0.9,0,0}
\definecolor{pgrey}{rgb}{0.46,0.45,0.48}

\lstset{language=Python,
  showspaces=false,
  numbers=left,
  showtabs=false,
  breaklines=true,
  showstringspaces=false,
  breakatwhitespace=true,
  commentstyle=\color{pgreen},
  keywordstyle=\color{pblue},
  stringstyle=\color{pred},
  basicstyle=\ttfamily,
  moredelim=[il][\textcolor{pgrey}]{$ $},
  moredelim=[is][\textcolor{pgrey}]{\%\%}{\%\%}
}



\hypersetup{
    colorlinks,
    citecolor=black,
    filecolor=black,
    linkcolor=black,
    urlcolor=black
}


% ------ HEADERS AND FOOTERS -----------
% \lhead{INFO-F403}
% \rhead{Project Report - Part 1}
%\pagestyle{fancy}
% \rfoot{\thepage}
%\cfoot{}
%\lfoot{Academic Year 2017-18}

\newcommand{\HRule}{\rule{\linewidth}{0.5mm}} %newcommand for cover page

\begin{document}

\begin{titlepage}
\begin{center}


\textsc{\LARGE universit\'e libre de bruxelles}\\[1.0cm]
\textsc{\Large D\'epartment d'Informatique}\\[1.5cm]

% Upper part of the page. The '~' is needed because \\
% only works if a paragraph has started.
% \includegraphics[width=0.3\textwidth]{image/ulblogo.jpg}~\\[1cm]

\textsc{
\large INFO-F404 \\
\Large  Real-Time Operating Systems
 \\[1cm]}
% Title
\HRule \\[0.7cm]

{ \huge \bfseries Project 1 – Audsley  \\[0.7cm] }

\HRule \\[2cm]

% Author and supervisor
\noindent
\begin{center} \large

\emph{Author:}\\
\Large Hakim \textsc{Boulahya}\\
Youcef \textsc{Bouharaoua}
\end{center}
\begin{center} \large

% \emph{Professor:} \\
% Gilles \textsc{Geeraerts}\\

\end{center}

\vfill

% Bottom of the page
{\large \today}

\end{center}
\end{titlepage}

\tableofcontents
\newpage

\section{Simulator}

\paragraph{Implementation choices}

The simulator of the single FTP simulator is implemented in a discret manner
from \texttt{start} to \texttt{stop}. For each time steps
we execute functions that will update the
current running job informations, the different requests (arrivals or deadlines)
and will store them in datasets. We choose not to print the different
actions during the simulation,
since the simulation is used for other commands such as
the plotter or the audsley algorithm. The output can be produced by
a call to the method \texttt{FTPSimulation.output()}.

\subsection{Periodicity of the requests}

\paragraph{}

There were two possible implementatisn to simulate the scheduler in a
interval given as a parameter. The first idea was to simulate the scheduler
between the feasability interval $[0, S_n + P)$, but it would be tidious
to do it like this, since we would have to find the corresponding interval
in $[0, S_n + P)$ of the $[start, stop)$ interval of the simulation.

\paragraph{Arrivals periodicity}

The important part of the simulation is to detect the arrivals \textit{i.e.}
the job requests and the deadlines of those jobs. We know that
the job requests for each task are periodic.
Let $\tau_i = (O_i, T_i, D_i, C_i)$ a periodic task. We know that each
job request of $\tau_i$ have a periodicity of $T_i$ with the first request
made at time $O_i$. Ce can detect if a request for $\tau_i$ occur at time
$t$ using the following formula:

\begin{equation}
  \label{arr_period}
  [(t - O_i) \ge 0] \;\; and \;\; [(t - O_i) \; mod \; T_i = 0]
\end{equation}

The left-side formula is the detection of the first request. If $t - O_i$
is bigger than 0,
then it means that it is possible that $\tau_i$ request a job.
The right-side formula is the detection of a request at time $t$. To detect
which job it is we can make an entire division of $t - O_i$ with
the period $T_i$. Figure \ref{fig:arrival_for} is a excerpt of the arrival
detection.

\begin{figure}[H]
    \begin{lstlisting}
offset, period = self.tasks[task_id][O], self.tasks[task_id][T]
cond = (t - offset >= 0) and ((t - offset) % period) == 0
job_id = (t - offset) // period  # module == 0 i.e. no decimals
return (job_id, True) if cond else (None, False)
    \end{lstlisting}
    \caption{Function of the arrival detection for a task at time $t$}
    \label{fig:arrival_for}
\end{figure}

\paragraph{Deadlines periodicity}

For each job requested job there is a deadline associated \textit{i.e.}
the deadline is also periodic. To detect if a deadline occur at time $t$
of a task $\tau_i$ we can use the same formula (\ref{arr_period}), and
testing the formula a time $t - D_i$. This method actually detect if
a job was requested, and we are a the requested time plus the deadline, the
it is the deadline of this job. Figure \ref{fig:deadline_for} shows
a excerpt of the deadline detection.

\begin{figure}[H]
    \begin{lstlisting}
deadline = self.tasks[task_id][D]
return self.is_arrival_for(t - deadline, task_id)
    \end{lstlisting}

    \caption{Function of the deadline detection for a task at time $t$}
    \label{fig:deadline_for}
\end{figure}


\subsection{Pending jobs}

\paragraph{} To be able to process requested jobs, the simulator uses a list
of list to handle pending jobs. A sublist per task exists. When a request
of a task occurs, the new job is added to the pending jobs sublist of the
task.
The sublists are ordered by task priority. Therefore the job to process
is the first job of the first sublist. If this list is empty, then
it process the first job of the second list, etc. When a job finished its
computation \textit{i.e.} has been processed $C_i$ time, then it is removed
from the pending jobs list. Figure \ref{fig:pending_jobs} shows the snippet
of the function that returns the job to be processed.

\begin{figure}[H]
    \begin{lstlisting}
for sub_jobs in self.pending_jobs:
    for job in sub_jobs:
        return job
return None
    \end{lstlisting}
    \caption{Job to be processed detection}
    \label{fig:pending_jobs}
\end{figure}

\subsection{Events}

\label{sec:events}

\paragraph{}

For each time steps, the simulator stores relevant informations that can
be used to output informations on the standard output or in a plot.
Those informations are stored in an object \texttt{Event}
which is composed of actions that occurs at time $t$:
\begin{itemize}
  \item The requests
  \item The deadlines (and the missed deadlines)
  \item The computed job
\end{itemize}

\subsection{Hard vs Soft simulation}

\paragraph{}


It is possible to configure the simulation to consider the deadlines as
soft or hard. If the soft configuration is chosen, the simulator will
not stop at deadline misses. If the hard configuration is chosen, the
simulator will stop at the first deadline miss (see section \ref{sec:howto}
to see how to run each configuration).

\subsection{Output}

\paragraph{}


The output production is a post process function. It has been implemented
like this to avoid printing while doing the simulation because the simulation
is used by other part of the project such as the implementation of the
Audsley algorithm or the Schedule plotter. The \texttt{output()} method
uses the events of the simulator (section \ref{sec:events}).


\paragraph{}
The simulator provides two approch to the output. The default
mode is the \textbf{request mode}. It provide an output as shown in Figure
\ref{fig:req_mode}. This mode output the requests,
such as arrivals or deadlines, as soon as they occur. So it does output
even if a job is processing during a block.
This mode has been implemented
to provide a more continuous approch to the simulation.

The second approch is the \textbf{preemptive mode}. As explained above
the process block of a job can be divide in multiple blocks if requests
occur during the process. To propose an output that characterize more
the preemptive proprety of a FTP simulator, the preemptive mode provide
a preemptive resuts as shown in Figure \ref{fig:pre_mode}.

\paragraph{}
Section \ref{sec:howto} explained how to run the different modes.




\begin{figure}[H]
  \begin{subfigure}{1\textwidth}
    \begin{lstlisting}
    \end{lstlisting}
      \centering
      \begin{lstlisting}[numbers=none]
                              0 5 10 5
                              0 3 20 2
      \end{lstlisting}
      \caption{Tasks set to output $(O_i, T_i, D_i, C_i)$}
  \end{subfigure}
  \begin{subfigure}{.5\textwidth}
      \centering
      \begin{lstlisting}[numbers=none]
      0: Arrival of job T1J1
      0: Arrival of job T2J1
      0-3: T1J1
      3: Arrival of job T2J2
      3-5: T1J1
      \end{lstlisting}
      \caption{Requests mode output}
      \label{fig:req_mode}
  \end{subfigure}
  \begin{subfigure}{.5\textwidth}
      \centering
      \begin{lstlisting}[numbers=none]
      0: Arrival of job T1J1
      0: Arrival of job T2J1
      0-5: T1J1
      3: Arrival of job T2J2
      \end{lstlisting}
      \caption{Preemptive mode output}
      \label{fig:pre_mode}
  \end{subfigure}
\end{figure}




- necessary to run the full simulation to be sure that it works

- Using the hyper period vs calculation with modulo

- explain output (preemptive non-preemptive)

- explain soft and hard simulation

- HowTo : simulation hard/soft, --premptive

\section{Plotter}

Show scheduler.png of tassk.txt 0 200 and audsley 0 400

\section{How to run}

\label{sec:howto}


\end{document}
