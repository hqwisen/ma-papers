\documentclass[letterpaper]{article}
\usepackage{natbib}
\usepackage[utf8]{inputenc}
\usepackage[margin=3.5cm]{geometry}
\usepackage{listings}
\usepackage{adjustbox}
\usepackage{xcolor}
\usepackage{verbatim}
\usepackage{graphicx}%images
\usepackage{fancyhdr}%for headers and footers
\usepackage{adjustbox}
\usepackage{verbatim}
\usepackage{listings}
\usepackage{fancyhdr}
\usepackage{multirow}
\usepackage{amsmath}
\usepackage{array}
\usepackage{mathpazo}
\usepackage{subcaption}
\usepackage{float}
\usepackage{csvsimple}
\usepackage{filecontents}
\usepackage{lscape}
\usepackage{afterpage}
\usepackage{hyperref}
\usepackage{inconsolata}
\usepackage{color}


\newenvironment{conditions}
  {\par\vspace{\abovedisplayskip}\noindent\begin{tabular}
  {>{$}l<{$} @{${}={}$} l}}
  {\end{tabular}\par\vspace{\belowdisplayskip}}


  \newenvironment{conditions_bis}
  {\par\vspace{\abovedisplayskip}\noindent\begin{tabular}
  {>{$}l<{$} @{${}\in{}$} l}}
  {\end{tabular}\par\vspace{\belowdisplayskip}}

\definecolor{pblue}{rgb}{0.13,0.13,1}
\definecolor{pgreen}{rgb}{0,0.5,0}
\definecolor{pred}{rgb}{0.9,0,0}
\definecolor{pgrey}{rgb}{0.46,0.45,0.48}

\lstset{language=Python,
  showspaces=false,
  numbers=left,
  showtabs=false,
  breaklines=true,
  showstringspaces=false,
  breakatwhitespace=true,
  commentstyle=\color{pgreen},
  keywordstyle=\color{pblue},
  stringstyle=\color{pred},
  basicstyle=\ttfamily,
  moredelim=[il][\textcolor{pgrey}]{$ $},
  moredelim=[is][\textcolor{pgrey}]{\%\%}{\%\%}
}



\hypersetup{
    colorlinks,
    citecolor=black,
    filecolor=black,
    linkcolor=black,
    urlcolor=black
}


% ------ HEADERS AND FOOTERS -----------
% \lhead{INFO-F403}
% \rhead{Project Report - Part 1}
%\pagestyle{fancy}
% \rfoot{\thepage}
%\cfoot{}
%\lfoot{Academic Year 2017-18}

\newcommand{\HRule}{\rule{\linewidth}{0.5mm}} %newcommand for cover page

\begin{document}

\begin{titlepage}
\begin{center}


\textsc{\LARGE universit\'e libre de bruxelles}\\[1.0cm]
\textsc{\Large D\'epartment d'Informatique}\\[1.5cm]

% Upper part of the page. The '~' is needed because \\
% only works if a paragraph has started.
\includegraphics[width=0.3\textwidth]{images/ulblogo.jpg}~\\[1cm]

\textsc{
\large INFO-?-??? \\
\Large  Prep. work for master thesis
 \\[1cm]}
% Title
\HRule \\[0.7cm]

{ \huge \bfseries Thesis - draft \#1 \\[0.7cm] }

\HRule \\[2cm]

% Author and supervisor
\noindent
\begin{center} \large

\emph{Author:}\\
\Large Hakim \textsc{Boulahya}\\
\end{center}
\begin{center} \large

% \emph{Professor:} \\
% Gilles \textsc{Geeraerts}\\

\end{center}

\vfill

% Bottom of the page
{\large \today}

\end{center}
\end{titlepage}

\tableofcontents
\newpage

\section{Introduction}


\section{Motivation and objective}

For the moment, the implementation to represent the data strutures in Acacia+
is specifically designed for a specific set. The idea is to propose a new
library implementation to provide an API that will implement important
data structures that are used in syntesis algorithms.

\paragraph{}

The objective of the thesis is to provide an efficient library to represent
antichain data structures and implement different operation. An final
goal is to be able to use this library within Aaron/Acacia.


\subsection{Related Work}

\subsubsection{Acacia+}

\subsubsection{AaPAL}


\section{Notions}

\subsection{Model checking and synthesis}

Model checking is the process to verify a software model with specifications.

Synthesis is the process to derive the system from specifications.

\subsection{Data structures}

\paragraph{Binary Decision Diagram}

\paragraph{Antichain}

BDD vs antichain

Antichain vs pseudo-antichain

\section{Questions}

* Maastricht library ?
* How to references ?
* What to impl (about operations) ?

* Is implementing LTL Rea. or Universality a final goal of the thesis ?

\bibliographystyle{plain}
\bibliography{thesis}

\end{document}
