\documentclass[letterpaper]{memoir}
% \usepackage{natbib}
\usepackage[utf8]{inputenc}
% \usepackage[margin=3.5cm]{geometry}
% \usepackage{listings}
% \usepackage{adjustbox}
% \usepackage{xcolor}
% \usepackage{verbatim}
% \usepackage{graphicx}%images
% \usepackage{fancyhdr}%for headers and footers
% \usepackage{adjustbox}
% \usepackage{verbatim}
% \usepackage{listings}
% \usepackage{fancyhdr}
% \usepackage{multirow}
% \usepackage{amsmath}
% \usepackage{array}
% \usepackage{mathpazo}
% \usepackage{subcaption}
% \usepackage{float}
% \usepackage{csvsimple}
% \usepackage{filecontents}
% \usepackage{lscape}
% \usepackage{afterpage}
% \usepackage{hyperref}
% \usepackage{inconsolata}
% \usepackage{color}
\usepackage{todonotes}

% \newenvironment{conditions}
%   {\par\vspace{\abovedisplayskip}\noindent\begin{tabular}
%   {>{$}l<{$} @{${}={}$} l}}
%   {\end{tabular}\par\vspace{\belowdisplayskip}}
%
%
%   \newenvironment{conditions_bis}
%   {\par\vspace{\abovedisplayskip}\noindent\begin{tabular}
%   {>{$}l<{$} @{${}\in{}$} l}}
%   {\end{tabular}\par\vspace{\belowdisplayskip}}
%
% \definecolor{pblue}{rgb}{0.13,0.13,1}
% \definecolor{pgreen}{rgb}{0,0.5,0}
% \definecolor{pred}{rgb}{0.9,0,0}
% \definecolor{pgrey}{rgb}{0.46,0.45,0.48}
%
% \lstset{language=Python,
%   showspaces=false,
%   numbers=left,
%   showtabs=false,
%   breaklines=true,
%   showstringspaces=false,
%   breakatwhitespace=true,
%   commentstyle=\color{pgreen},
%   keywordstyle=\color{pblue},
%   stringstyle=\color{pred},
%   basicstyle=\ttfamily,
%   moredelim=[il][\textcolor{pgrey}]{$ $},
%   moredelim=[is][\textcolor{pgrey}]{\%\%}{\%\%}
% }
%
%
%
% \hypersetup{
%     colorlinks,
%     citecolor=black,
%     filecolor=black,
%     linkcolor=black,
%     urlcolor=black
% }
%
%
% % ------ HEADERS AND FOOTERS -----------
% % \lhead{INFO-F403}
% % \rhead{Project Report - Part 1}
% %\pagestyle{fancy}
% % \rfoot{\thepage}
% %\cfoot{}
% %\lfoot{Academic Year 2017-18}
%
% \newcommand{\HRule}{\rule{\linewidth}{0.5mm}} %newcommand for cover page


%\begin{titlepage}

% \begin{center}

%
% \textsc{\LARGE universit\'e libre de bruxelles}\\[1.0cm]
% \textsc{\Large D\'epartment d'Informatique}\\[1.5cm]
%
% % Upper part of the page. The '~' is needed because \\
% % only works if a paragraph has started.
% \includegraphics[width=0.3\textwidth]{images/ulblogo.jpg}~\\[1cm]
%
% \textsc{
% \large MEMO-F-403 \\
% \Large  Preparatory work for the master thesis
%  \\[1cm]}
% % Title
% \HRule \\[0.7cm]
%
% { \huge \bfseries Thesis - draft \#1 \\[0.7cm] }
%
% \HRule \\[2cm]
%
% % Author and supervisor
% \noindent
% \begin{center} \large
%
% \emph{Author:}\\
% \Large Hakim \textsc{Boulahya}\\
% \end{center}
% \begin{center} \large
%
% \emph{Supervisor:} \\
% Emmanuel \textsc{Filiot} \\

% \end{center}
%
% \vfill
%
% % Bottom of the page
% {\large \today}
%
% \end{center}
%\end{titlepage}


\title{(Pre)Thesis draft}
\author{Hakim Boulahya}

\begin{document}

\maketitle

\tableofcontents

\listoftodos

\chapter{Introduction}


\section{Motivation}

Automata theory is used in various field in Computer Science, especially
in Computer-Aided Verification.

\todo{cite the original papers of those problems}
Problems such as synthesis \cite{ltl_rea}
or the universality problem \cite{ant_univers}
\todo{Ref for complexity of univers. pbl}
has shown to be PSPACE-complete \cite{bsp, ant_univers}.

More efficient algorithms to resolve those problems have been implemented
using antichain based-algorithms. Antichains are data structures that allow
to represent a partial order sets, in a more compact way.
\todo{Talk about
the problems, complexity and alternative (Safra vs antichain)}.


\section{Objective}

The goal of this thesis is to provide an efficient implementation of
data structures used in those algorithms, especially antichain. We will
mainly focus on implementing antichain-related data structure and provide
a library to be used in different tool such as Acacia+, Owl.

The first, and main, objective is to implement an Antichain object
in Java to be used in Owl. Then if possible used this implementation
for other tool such as Acacia+, by either providing bindings or other.

As we mainly focus on effiency, it could be interesting to use a C
implementation and provide binding to a Java class.
\todo{This small paragraph is an open discussion}

\section{Related work}

\paragraph{}

AaPAL library is a that was implemented in the context of Bohy's PhD thesis
to provide an antichain library and be able to implement the antichains
based algorithm for the synthesis problem.

\paragraph{}

Java already provide built-in implementation for Set.
\todo{Includes limitation of Java built-in and different possible solution
for antichains found on stack overflow}

\chapter{Data Structures}

In this section, we will provide formal definitions of the data
structures that we will implement. We recall the notion of binary relations
and important propreties of such relations.
We then define partially ordered set, totally order set and closed set.
Finally we give a formal definition for antichains.

\paragraph{}

The definitions and examples for this section are based on \cite{bohy_phd}.


\section{Binary relations}

\paragraph{}

A binary relation for an arbitrary set $S$ is
a set of pair $R \subseteq S \times S$.
There are five important properties: reflexitivity, transitivity,
symmetry, antisymmetry and total.

\paragraph{}

A relation $R$ on $S$ is said to be:

\begin{itemize}
    \item Reflexive:
    iff $\forall s \in S$ it holds that $(s, s) \in R$
    \item Transitive:
    iff $\forall s_1, s_2, s_3 \in S$,
    if ($s_1, s_2) \in R$ and $(s_2, s_3) \in R$
    then it holds that $(s_1, s_3) \in R$
    \item Symmetric: iff $(s_1, s_2) \in R$ then $(s_2, s_1) \in R$.
    \item Antisymmetric: iff $(s_1, s_2) \in R$
    and $(s_2, s_1) \in R$ then $s_1 = s_2$
    \item Total: iff $\forall s_1, s_2 \in S$ then $(s_1, s_2) \in R$
    or $(s_2, s_1) \in R$\todo{is this Total def correct ?}

\end{itemize}

\paragraph{Orders}

A \textit{partial order} is a binary relation that is \textit{reflexive},
\textit{transitive} and \textit{antisymmetric}. We note a
partial order relation by $R$.
We note $s_1 R  s_2$ to show the belonging of
a binary relation to a partial order, which is equivalent
to $(s_1, s_2) \in \ R$.
A \textit{total order} is a partial order that is \textit{total}.

\section{Partially ordered set}

\paragraph{}

An arbitrary set $S$ associated with a partial order $\preceq$
is called a \textit{partially ordered set} or \textit{poset}.
It is denoted by the pair $\langle S, \preceq \rangle$.

\paragraph{Comparable}

Let $s_1, s_2 \in S$ and $\langle S, \preceq \rangle$ a poset.
The two elementes $s_1$ and $s_2$ are called \textit{comparable} if either
$s_1 \preceq s_2$ or $s_2 \preceq s_1$. If neither of those two comparaisons
are correct, then $s_1$ and $s_2$ are called \textit{uncomparable}.


\paragraph{Bounds}

\paragraph{Lattices}

A poset where every elements has a LUB and GLB is called a lattice.

lower semilattice: all elements have LUB.
upper semilattice: all elements have GLB.


\section{Antichains and pseudo-antichains}

\paragraph{Closed sets}

A closed set is a set $L \subseteq S$
of a lower semilattice $\langle S, \preceq \rangle$
where $\forall l \in L$ we have that $\forall s \in S$ such that
$s \preceq l$, then $s \in L$.

\paragraph{Closure}



\todo{give definition of (un)comparable using partial order definition}

\todo{Is this total order affirmation correct ?}
In this thesis we are more interested in partial ordered sets as
total order sets can be easily implemented as lists.

\paragraph{Ordered sets}

Partial/Order sets

Lattice (semi upper and lower)

Closed sets

Antichain

Includes propreties to implement



\chapter{Implementation}


\section{Summary of objectives}

\paragraph{}


The main focus of the thesis is to be able to provide an efficient
implementation of antichains and pseudo-antichains in \texttt{Java}.
The first step is to provide an interface for the different operations that
can be applied to antichains. We then give a description of the implementation.
Antichains provide a way to represent
in a compact way partially ordered set that are closed. Pseudo-antichains
are an extension of antichains and provide a compact way to represent
partially ordered sets. Pseudo-antichains does not specifically require
closed set.



\section{Existing implementation}




\todo{Everything below is a fast/draft notes}

\section{Motivation and objective}

For the moment, the implementation to represent the data strutures in
 Acacia+
is specifically designed for a specific set. The idea is to propose a new
library implementation to provide an API that will implement important
data structures that are used in syntesis algorithms.

\paragraph{}

The objective of the thesis is to provide an efficient library to represent
antichain data structures and implement different operation. An final
goal is to be able to use this library within Aaron/Acacia.


* Impl. datastrcture antichain in Java
* Theoritacal context: synthesis,
unverversality and automata theory known problem
* Practical context: Owl, Acacia+ and AaPAL

* Why interesting: More efficient than CTL symbolic with BDD

\subsection{Related Work}

* Java Built-in IMPL
* AaPAL
* Stackoverflow
* Acacia ?

\section{Notions}

\subsection{Model checking and synthesis}

Model checking is the process to verify a software model with
specifications.

Synthesis is the process to derive the system from specifications.

\cite{bohyphd}

\cite{ltl_rea}
\subsection{Data structures}

\paragraph{Binary Decision Diagram}

\paragraph{Antichain}

BDD vs antichain

Antichain vs pseudo-antichain

\section{Questions}

* Maastricht library ? \cite{acacia} \cite{aapal}
* How to references ?
* What to impl (about operations) ?

* Is implementing LTL Rea. or Universality a final goal of the thesis ?

\bibliographystyle{alpha}
\bibliography{thesis}

\end{document}
