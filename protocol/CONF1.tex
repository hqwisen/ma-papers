\documentclass[10pt,a4paper]{article}
% \usepackage{natbib}

\usepackage[utf8]{inputenc}
\usepackage{mathpazo}
\usepackage{amsmath,amsfonts,amssymb,amsthm}
\usepackage{mathtools}
\usepackage{todonotes}


\title{INFO-F-??? - Conference Report #1 \\
Atos Wordline}
\author{Université Libre de Bruxelles \\
\\ Hakim Boulahya}


\begin{document}

\maketitle

% \begin{abstract}
% \end{abstract}

\section{Introduction}

During the conference of 28 March, Cedric Meuter
talk about his activity in Atos Worldline company.
The subject of the conference was the use of
cryptographic technology within cashless banking payment.

In this paper, we will propose a summary of the content of the
conference. We will first describe fundamentals of security.
Then we will propose a description of the important
cryptographic protocol uses in the doman. Then an important
part of the job in banking is to handle the keys used
during the communication of the devices during a payment,
we will then describe the key management.
Finally we will propose some interesting details about
engineering in security.


\section{Fundamentals of security}

It is important for a company to analyse the risk
of the infrastrucutre deployed. Two important questions must
be asked for (1) How likely an attacker will try to compromise
you and acquire your data (2) What's the impact of the compromisation
of your data. A good example of of an attack that highlights
the importance of such risk analysis is the Sony Playstation Network
Attack \cite{raiu_cyber-threat_2012}. Sony acknowledge that
personal informations were stolen, including credit cards number.
There are three important fundamentals of security:
(1) Confidentiality (2) Integrity (3) Authenticity.

\section{Cryptographic protocols}

The Payment Card Industry (PCI) must rely on standard to make
sure that the transactions follow the fundamentals of security.
Those standards used by the PCI are defined by different institution, such
as NIST, FIPS or ANSI. An important protocol used in terminal payment
is the DUKPT protocol defined in \cite{dukpt_std}.

\paragraph{DUKPT}

DUKPT or Derived Unique Key Per Transaction, is at it's named
says to generate a unique key per transaction.
The goal of such a protocol, if a single transaction is compromised,
it will not compromised the all infrastructure.
To do so, it is necessary to have a centralized server, usually
a HSM (we discuss in more details HSM in section \ref{sec:hsm}).


Trouver des ref pour DUKPT explnation and HSM

Give more technicals details about DUKPT and HSM
\section{Key Management}

\label{sec:hsù}


\bibliographystyle{alpha}
\bibliography{CONF1}

\end{document}
