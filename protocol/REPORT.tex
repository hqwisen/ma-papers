\documentclass[11pt,a4paper]{article}
% \usepackage{natbib}

\usepackage[utf8]{inputenc}
\usepackage{mathpazo}
\usepackage{amsmath,amsfonts,amssymb,amsthm}
\usepackage{mathtools}
\usepackage{todonotes}

\usepackage[margin=3cm]{geometry}



\title{INFO-F-514 - Year Report}
\author{Université Libre de Bruxelles \\
\\ Hakim Boulahya}


\begin{document}

\maketitle
\tableofcontents

\newpage

% \begin{abstract}
% \end{abstract}

\section{Secure computation}

\subsection{Introduction}

In this paper we will propose formal definitions of secure
computation, also refered as secure multiparty computation.
We will also gives known example in the litterature
that are defined following the logical of secure computation.
We will then recall
state of the art techniques and protocols that allow
to resolve secure computations.

\subsection{Secure computation}

\paragraph{}

A secure multi-party computation problem, is a problem where a computation,
or a result, must be computed but the input that each party must used
is confidential and not share between all parties. Such problem
can be defined as a function $f(\cdot)$, that takes $n$ parameters.
The idea is to be able to compute the function $f(x_0, .., x_n)$
where the input $x_i$ can only be accessed by the party $i$.
The final results is accessible to everyone.

\paragraph{}

The first secure computation problem was first introduced by
Yao in 1985 \cite{yao_protocols_1982}, with the millionnaires problem.
This problem is a secure two-party computation, a subproblem
of multi-party computation problem. Unlike other cryptographic
protocols, the malicious behaviour come from the
participant in the exchange. Indeed in secure computation
we can define two party behaviour. A \textit{semi-honest adversary}
is a party in the protocol that will always follow the steps
that must be performed as stated in the protocol. That is, a semi-honest
adversary will always send well-formed messages. It is not
fully honest because such adversaries will try to learn other
participants secrets by analyzing their protocol messages.
A \textit{malicious adversary} can behave in the worst way possible.
That is it can deviate from the protocol, send mal-formed message,
and can use any other possible way to find the other parties secrets.
When implementing such protocols, it is necessary to take in consideration
the behaviour of the parties, that is the security of the protocol
may rely on the honesty of the participants.

\subsection{Oblivious Transfer}

\subsubsection{Definition and variants}

\paragraph{}


The Oblivious Transfer introduced by Rabin during
in 1982 \cite{rabin_how_nodate}. The Oblivious Transfer has many application
and has been first introduced has a protocol to resolve the Exchange Of
Secret problem.
The oblivious transfer protocol is defined as follow:
a sender want to send a message to a receiver, but it
must not be able to tell if the receiver got the message, that
is there is a probability of $\frac{1}{2}$, that the message
has been sent to the receiver.

\paragraph{}


In the context of secure computation,
The 1-out-of-2 Oblivious Transfer ($OT^1_2$),
an another approach to the original Oblivious Transfer, is used.
It is the problem
that for a sender and a receiver, one of two message must be sent
from the receiver to the sender. The message receive can be chosen
by the receiver. Two constraints are that the sender must never know
which message has been chosen, and the receiver must not know
the content of the other message.

\paragraph{}


There exists also
1-out-of-n Oblivious Transfer ($OT^1_n$) is an extension of $OT^1_2$,
where the sender has $n$ messages to send and the receiver must choose
one of them. Those two protocols are theorically equivalent
has proven in \cite{goos_equivalence_1988, goos_foundations_1998}.


\subsubsection{1-out-of-2 Oblivious Transfer protocol}

\label{sec:ot}

On possible protocol for the $OT^1_2$ problem is by using a pair of
key using the RSA protocol, first proposed in \cite{even_randomized_1985}.
Let Alice be the sender and Bob the receiver.
Alice has two messages $m_0,m_1$, and in addition to that
a public RSA key $(e, d, n)$. The protocol is a multi-step
communication between the two parties using the RSA public key
of Alice.

\paragraph{}

The first step is for Alice
to send the public key and two random values, that is
the public key $(e, n)$ and two random values $x_0, x_1$
contains in the domain $[1, n-1]$. Now that Bob has those inputs,
he will generates on his side two other random values.
The first one is the bit $b$ which value is either $0$ or $1$,
and is used to choose which random inputs received from Alice,
that is $x_b$ would be either $x_0$ or $x_1$. The second
generated random value of Bob is a value $k$ in the domain
$[1, n-1]$.

\paragraph{}


The second step is for Bob to return his response. Since
we don't want Alice to know which value has been chosen, Bob
will encrypt the value $x_b$ by blinding it using the random value $k$
that he generated. That is Bob will send to Alice
the value $v = (x_b + k^e) \mod n$.
Upon receival of $v$, Alice will decrypt $v$ two times, by removing
the random values. That is, Alice will have two values $k_0, k_1$
where $k_i = (v - x_i)^d \mod n$.

\paragraph{}

Finally, the last step if for Alice to send back the real message.
Since $v$ was blinded by Bob
with the value $k$, Alice doesn't know which random $x_i$ has been chosen.
By computing the two $k_i$ based on both values, one of them
will be identical to the $k$ value of Bob. The last inputs that
Alice will send to Bob are the two messages $m'_0, m'_1$ where
$m'_i = m_i + k_i$. Upon receival, Bob will have to decrypt
the message with $k$, that is $m_b = m'_b - k$. Since Bob
only has the $k_i$ value associate to his message, he will not be
able to decrypt the other message.

\subsection{An application: Yao's millionaires problem}

\paragraph{}

Oblivious Transfer provides a protocol  to share inputs without
giving to much informations to the other parties. It is
still necessary to provide a protocol that will
compute the function for the secure computation, and share
the results among parties. With this objective, we will
focus on a well known problem in secure computation, that
is Yao's millionaires problem.

\paragraph{}

The millionaires problem introduced by Yao in
\cite{yao_protocols_1982}, is the problem that for two millionaires
they both want to know which one of them is the richer, but they
don't want to know the difference. In this problem, the computation
function is the usual comparison $<$, and the inputs are the incomes of
the individuals. We will present two solutions of the problem,
where both use Oblivious Transfer protocol to securely exchange
informations. The first solution is Iaonnidis and Grama
 \cite{ioannidis_efficient_2003} where
they propose a solution using 1-out-of-2 oblivious transfer that construct
parallel computation of the comparison, using XOR operations.
The second solution
is the one proposed by Lin and Tzeng
\cite{hutchison_efficient_2005}, that uses homomorphic encryption
scheme.


\subsubsection{Ioannidis and Ananth solution}

Ioannidis and Ananth protocol to
resolve
the millionaires problem \cite{ioannidis_efficient_2003}
is divided in five major steps, and make
used of the 1-out-of-2 oblivious transfer protocol.

\paragraph{}

Alice and Bob wants to know which one is richer that the other.
Let $a$ be the number of millions she posses and $b$ the amount
of Bob. Before the computation, they first both agree
that $a, b < d$, for $d \in \mathbb{N}$.

\paragraph{Protocol}

The goal of this protocol is for Alice to first create
a matrix where all elements are values of $k$ bits, where
$k$ is the size of the RSA key of Alice, used in the Oblivous Transfer.
The first step for alice is to set the matrix to specific values
depending on the key size $k$, and bits of her value $a$.
The second major steps is to generate $d - 1$ $k$ bits random
numbers $S_i$. The create another value $S_d$ so that all the
bits are random except for the the last, that is $k-1$ and $k-2$.
Those two last bits are set using the bitwise XOR operation,
based on the $S_i$ random values and the generated matrix.
Then a rotation of the elements in the two first column
is made, by rotating the the cell with the second random
value generated by Alice at the beginning. The resulting
matrix is send to Bob, which will resend using
the oblivious transfer protocol for each line $i$, the value
at the column $b_i + 1$, where $b_i$ is the $i^{th}$ bit of $b$.
At reception, Alice will rotate the $S_i$ values using
her generated random numbers, and send back the result to Bob.
This will allow Bob to scan the value from the matrix at indices $b_i + 1$
(as explained above) and the $S_i$ rotated values received from Alice.
The scan should reveal a large sequence of zero bits. If
the bit to the right of the sequence is equals to $1$ then
$a \ge b$, otherwise $a < b$.


\subsubsection{Lin and Tzend solution}

\paragraph{Preliminaries}


Their protocol is based on multiplicative homomorphic encryption
scheme, and on binary encodings, that are 0-encoding and 1-enconding.

A multiplicative homomorphic encryption function is function
that when computed with a specific operator $\circ$, the results is equal
to the encryption of the multiplication of the message, that is:

$$
E(m_1) \circ E(m_2) = E(m_1 \times m_2)
$$

Regarding the encoding, a 0-encoding for a binary number $b$ of length
$n$ is a set $S_s^0$ of binary strings such that $S_s^0 = \{s_n s_{n-1}...s_{i+1}
1|s_i=0,1 \le i \le n\}$. A 1-encoding of $s$ is the set
$S_s^1 = \{s_n s_{n-1}...s_{i}|s_i=1,1 \le i \le n\}$. The goal
of using such encoding is to be able to reduce the millionaires problem
to the set intersection problem. Indeed, Let $S_x^1, S_y^0$ be
respectively the 1-encoding of a binary number $x$ and the 0-encoding
of a binary number $y$, $x \le y$ if and only if $S_x^1 \cap S_y^0 = \emptyset$.

\paragraph{Protocol} The secure computation protocol is divided
in 3 steps. Let $a$ be Alice private input, and $b$ Bob privates input,
both binary number of size $n$.
They want to know if $a < b$ without
sharing their inputs. Alice will choose an homomorphic encryption
scheme $(G, E, D)$, respectively a random generator, an encryption function
and a decryption function.
They will do the following exchange:
\begin{enumerate}
    \item  Alice:
    Generate a key pair from $G$ $(pk, sk)$ for $E$ and $D$.
    Generate a square matrix $T$ of size $n$ where the elements
    at positions $(a_i, i) = E(1)$ and $T(\overline{x}_i, i) = E(r_i)$,
    where $r_i$ is a random number generated by $G$. This matrix
    is send to Bob.
    \item Bob: For all elements $s$ in $S_b^0$, he will compute
    a value $c_s = T[t_n][n] \circ T[t_{n-1}][n-1]...\circ T[t_i][i]$.
    Then he will scalarize and randomly permute $c_s$ values.
    Send to Alice all $c_s$ values.
    \item Alice: Decrypt $D(c_i) = m_i$ for all $1 \le i \le n$, and
    determize $x < y$ if and only if any $m_i = 1$.
\end{enumerate}


\subsection{Conclusion}

\paragraph{}


We have presented secure computation and the different protocols
that were implemented to resolve such problem. We summarize two solutions,
one using the oblivious transfer protocol, and one using
homomorphic encryption scheme. There exists other solutions for
the protocol, including the one introduced in the original paper
\cite{yao_protocols_1982}
that presented secure computation. Because the goal
of the paper is to introduce secure computation and summarize
solution to the problem, we only presented some of the \textit{recent}
efficient solutions.

\newpage

\section{Conference \#1: FAB Framework}

\subsection{Introduction}

\paragraph{}


During the conference of 28 March, Mohammed El Kandri presented
Fast Access Blockchain, a framework that allow to overcome
the scalability of blockchain technology. In this paper, we
will first introduce the blockchain and the bitcoin applications.
We will the explain what is the scalability problem. Finally
we will present the Fast Access Blockchain framework and how
it propose to deal with the salability problem.

\subsection{Blockchain}

\paragraph{}


Blockchain is a technology that allow to decentralize a
system to avoid using a thrusted third-party, such as a bank.
The blockchain technology was first introduced by
Satoshi Nakamoto \cite{bitcoin} within the framework
of the Bitcoin blockchain, a decentralized network
that allow to exchange coins in form of transactions.

\paragraph{}

The different techniques first presented in \cite{bitcoin}, are
a combination of different cryptographic functions that allow
this massive scale decentralization. The goal is to make
sure that a transaction that is acknowledge by the network,
can be thrusted where no users can be thrusted.
The different cryptographic protocols used allow to insure
the will of a user to send his coins, avoid double-spending and
insure that nobody in the network can change any past transactions.


\paragraph{}

Let's summarize the Bitcoin protocol. A coin is represented by
a sequence of transactions. When Alice wants to send Bob a coin,
she musts signed a SHA-256 hash, using the ECSDA digital signature
algorithm, composed of the last transaction of the coin
and the one that she wants to create. This new transaction
will be send to the decentralized network,
usually a peer-to-peer network. Miners in the network
will provide a computational work called proof-of-work.
The goal of the proof-of-work is to mint a block, a
file containing multiple transactions, using a cost-function
of the hashcash \cite{hashcash} (the proof-of-work) algorithm.
A cost-function should be easy to verify but hard to compute.
This mechanism allow to create a Blockchain, that can be seen
as a linked list of blocks, where the proof-of-work make it
hard to edit the blockchain because each added block
is linked to the previous on in the blockchain.

\subsection{Scalability}

\paragraph{}
One problem that arise using this protocol is the scalability problem.
Because the number of transactions per block is fixed
(the block creation frequency is also fixed)
and the
proof-of-work takes time, the amount of transactions
that the bitcoin decentralized network can process is limited.

\paragraph{}

The debit of transaction is a major concern of large scale
entreprise. In order to be able to use blockchain technology
for different applications, such as banking, real estate, energy
and others industry, it is important to be able
to process multiple transactions concurrently.
What has been proposed during the conference is the Fast Access Blockchain.
A \textit{public blockchain which overcomes the scalability challenge}.
The challenge is stated as following: keeping the characteristics
of a blockchain that is, permissionless, decentralized and open,
and allowing multiple transactions to be acknowledge at the same time.

\subsection{FAB Overview}

The solution process by Fast Access Blockchain presented during
the conference and by the developers
\cite{fabmedium} is to divide the network with 3 main components:
a foundation chain, additional chains, called annex chains and an
open storage architecture, which should have a theoritical potential
to process a million transactions, and resolve the scalability problem.

\paragraph{Foundation}

The foundation chain, is the core of the system and the functionnality
is to contain a root ledger, process the transactions and take
the final decision. It is the core of the decentralized architecture.
A technical implementation provided by FAB within the foundation blockchain
is a tool called KanBan. The main goal of KanBan is to avoid the foundation
blockchain to be overloaded, and to add scalability to the
processing, that is redirecting some transactions to the
annex chains.

\paragraph{Annex chains}

The annex chains are multiple blockchains that process the majority
of the transactions, through the KanBan. It allows for
business to join the public blockchain, therefore it means that
an annex chain is used for one specific business applications.
It is used with the SCAR (Smart Contract Address Router) tool, that is
used to execute the transactions between the annex itself
and the externals, that is the KanBan component.

\paragraph{OSA}

The last component, the Open Storage, is used to, as the name states, to
store all the public data of the blockchains, and allows fast access
to it. The data stored are the one stored in the public blockchain
and are already very thrustworthy, as it depends on the KanBan tool.

\newpage

\section{Conference \#2: Empowering the cashless society}

\subsection{Introduction}

During the conference of 28 March, Cedric Meuter
talk about his activity in Atos Worldline company.
The subject of the conference was the use of
cryptographic technology within cashless banking payment.
In this paper, we will propose a summary of the content of the
conference. We will first describe fundamentals of security.
Then we will propose a description of the important
cryptographic protocol uses in the domain. Then an important
part of the job in banking is to handle the keys used
during the communication of the devices during a payment,
we will then describe the key management as presented
by Atom Worldline company.

\subsection{Fundamentals of security}

It is important for a company to analyse the risk
of the infrastrucutre deployed. Two important questions must
be asked for (1) How likely an attacker will try to compromise
you and acquire your data (2) What's the impact of the compromisation
of your data. A good example of of an attack that highlights
the importance of such risk analysis is the Sony Playstation Network
Attack \cite{raiu_cyber-threat_2012}. Sony acknowledge that
personal informations were stolen, including credit cards number.
There are three important fundamentals of security:
(1) Confidentiality (2) Integrity (3) Authenticity.
With this informations, the goal of the protocols presented
in the next section is to respect those fundamentals.

\subsection{Cryptographic protocols and key management}

The Payment Card Industry (PCI) must rely on standard to make
sure that the transactions follow the fundamentals of security.
Those standards used by the PCI are defined by different institution, such
as NIST, FIPS or ANSI. An important protocol used in terminal payment
is the DUKPT protocol defined in
the ANSI X9.24 standard \cite{ansi_x924}.

\subsubsection{DUKPT}

DUKPT or Derived Unique Key Per Transaction, is at it's named
says to generate a unique key per transaction.
The goal of such a protocol, if a single transaction is compromised,
it will not compromised the all infrastructure.
To do so, it is necessary to have a centralized server, usually
a Hardware Security Module, a centralized server that stores
the keys and provides cryptographic functions.
The algorithms need a secret key,
called a Base Derivation Key or BDK, to be able to generate
a key for each terminal. Storing the BDK is very costly because
if they are compromised, they can compromised a lot of devices.
Therefore store new keys, developer must request to add a new
key, and this will be done during a Key Ceremony, that will
add the key to the HSM.

\paragraph{}

Each terminal as a secret key, based on the serial number,
generate at factory by the HSM. To encrypt a transaction and process
secret information, such as a PIN code, the terminal generate a number
of key, and each key will only be used for only one transactions.
Then the terminal send the encrypted data and it's serial number to the HSM.
The HSM will be able to retrieve the key from the terminal, and
regenerate the key used for the transaction, since it provides
the cryptographic functions, and the
key of the terminal was generated by the HSM.


\subsubsection{Hardware Security Modules}


\label{sec:hsm}

The HSM is the devices (server), that allows to generate
key for the terminals, and provides crypto-functions to handle
the payements. The HSM stores multiple Base Deriviation Key, and
with those will be able to generate per terminal keys,
called Initial Key (IK). Those keys are loaded in factory
in each terminal and will allow the terminals to generated
the Trasanctions Keys.


\bibliographystyle{alpha}
\bibliography{REPORT}

\end{document}
