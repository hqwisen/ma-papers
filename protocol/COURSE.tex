\documentclass[10pt,a4paper]{article}
% \usepackage{natbib}

\usepackage[utf8]{inputenc}
\usepackage{mathpazo}
\usepackage{amsmath,amsfonts,amssymb,amsthm}
\usepackage{mathtools}
\usepackage{todonotes}


\title{INFO-F-??? -     Course Report \\
Secure computation}
\author{Université Libre de Bruxelles \\
\\ Hakim Boulahya}


\begin{document}

\maketitle

% \begin{abstract}
% \end{abstract}

\section{Introduction}

In this paper we will propose formal definitions of secure
computation, also refered as secure multiparty computation.
We will also gives known example in the litterature
that are defined following the logical of secure computation.
We will then gives techniques and protocols that allow
to resolve secure computations.

\section{Secure computation}

A secure multi-part computation problem, is a problem where a computation,
or a result, must be computed but the input that each party must used
is confidential and not share between all parties. Such problem
can be defined as a function $f(\cdot)$, that takes $n$ parameters.
The idea is to be able to compute the function $f(x_0, .., x_n)$
where the input $x_i$ can only be accessed by the party $i$.
The final results is accessible to everyone.
\todo{cite f(.)}

\section{Problems}

\paragraph{}


There exists multiple problems that used the secure computation definition.
For example the millionaires problem, is the problem that for two millionaires
they both want to know which one of them is the richer, but they
don't want to know the difference. In this problem, the computation
function is the usual comparison $<$, and the inputs are the incomes of
the individuals.


\paragraph{}
Another problem is the Oblivious Transfer introduced by Rabin during
in 1982 \cite{rabin}.


\todo{cite mill problem}

\bibliographystyle{alpha}
\bibliography{CONF2}

\end{document}
