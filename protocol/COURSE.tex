\documentclass[10pt,a4paper]{article}
% \usepackage{natbib}

\usepackage[utf8]{inputenc}
\usepackage{mathpazo}
\usepackage{amsmath,amsfonts,amssymb,amsthm}
\usepackage{mathtools}
\usepackage{todonotes}


\title{INFO-F-??? -     Course Report \\
Secure computation}
\author{Université Libre de Bruxelles \\
\\ Hakim Boulahya}


\begin{document}

\maketitle

% \begin{abstract}
% \end{abstract}

\section{Introduction}

In this paper we will propose formal definitions of secure
computation, also refered as secure multiparty computation.
We will also gives known example in the litterature
that are defined following the logical of secure computation.
We will then gives techniques and protocols that allow
to resolve secure computations.

\section{Secure computation}





\bibliographystyle{alpha}
\bibliography{CONF2}

\end{document}
